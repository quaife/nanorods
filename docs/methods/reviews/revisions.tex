\documentclass[11pt]{article}


\usepackage{fullpage}
\usepackage{todonotes}
\usepackage{amsmath,amsfonts,amssymb,stmaryrd}
\usepackage{color}
\newcommand{\comment}[1]{{\color{blue} #1}}

\begin{document}
Thank you for reading ....

\section*{Reviewer 1}
\noindent
\comment{In section 3.2, it says that one Lagrange multiplier is used.
Are you using one contact volume constraint among all particles or one
contact volume constraint per contact region?}
\begin{itemize}
  \item The reviewer is correct that there is a single Lagrange
    multiplier for each pair of colliding bodies.  The text has been
    modified in Section 3.2.
\end{itemize}

\noindent
\comment{The locally implicit time stepping breaks on the Taylor-Green
  flow even with dt = $1e-8$. You can also decrease the time step size
  with globally implicit time stepping without using contact constraint.
  Is there a stable time step size comparison between globally implicit
  time stepping without contact constraint and globally implicit time
  stepping with contact constraint? What is the ratio between time spent
  on the NCP solve stage and the time spent on globally implicit time
  stepping without NCP solve?}
\begin{itemize}
  \item The globally implicit time stepper can be used without the
    contact constraint with an acceptable time step size.  While this
    results in a savings because no NCP solvers are required, it results
    in two additional computational costs.  First, more time steps must
    be taken.  Second, because the bodies come closer together,
    additional GMRES iterations are required to perform each time step.
    We describe this in detail in the Taylor-Green section.

  \item We quantify the additional cost of NCP as a percentage of the
    total computational effort
\end{itemize}

\noindent
\comment{Typo on equation 7e, the torque $L$.}
\begin{itemize}
  \item This is corrected
\end{itemize}

\noindent
\comment{Typos around page 8 line 27 and line 34 '$d_j$' in Rotlet and the
torque $L$?}
\begin{itemize}
  \item There were several typos in these equations.  The $\mathbf{d}_j$
    should have been $\mathbf{c}_j$ since these correspond to the center
    of the rigid bodies or the interior elements of the boundary of the
    solid walls.  The typo for the torque $L$ is also corrected.
\end{itemize}


%%%%%%%%%%%%%%%%%%%%%%%%%%%%%%%%%%%%%%%%%%%%%%%%%%%%%%%%%%%%%%%%%%%%%%%%%%%%%%
\section*{Reviewer 2}
\noindent
\comment{The introduction opens with a discussion about rods and fibers
  and the average stress tensor in Eq.~(1). This is not extended any
  further in the text or the results section. At its current state, it
  seems unrelated to the text. After only reading the introduction, one
  expects to see some analysis related to fibers and effective
viscosity, which is entirely missing from the text. Please revise the
introduction to reflect the current work.}
\begin{itemize}
  \item \todo[inline]{Bryan: Tone down the intro so that it reads more
    as a motivation for the work that we want to eventually do.}
\end{itemize}


\noindent
\comment{The contribution of the paper is declared to be time stepping
  strategy, going from locally-implicit to globally implicit. At the
  face of it, this does not seem to be novel. This makes the interaction
  matrix fully dense and more costly to apply, but seems trivial to
  implement otherwise. For instance, Malhotra et al.~(cited by the
authors) use a similar scheme in 3D.

Later in that section, it is mentioned that ``we use our new time
stepping to examine the rheological properties of dense rigid body
suspensions." However, there is no in-depth discussion about these and
the results section briefly goes over some example.

In summary, if this is a ``method paper", I fail to see any significant
mathematical/computational contributions. If it is an ``application
paper", it lacks depth in its analysis.}
\begin{itemize}
  \item \todo[inline]{Bryan: Malholtra et al.~consider deformable bodies
    which is less challenging.}
  \item \todo[inline]{Bryan: We do compute effective viscosities and
    alignment angles which are both related to the statistical and
    rheological properties of the flow.}
\end{itemize}

\noindent
\comment{In related work and then later in the text, authors mention
  they use a block-diagonal preconditioner. However, when particles are
  close to touching, the main source of ill-conditioning is the
  interaction between those two particles and not the self interaction.
  I do not believe that the block-diagonal preconditioner is of any
  (considerable) help in this setting.  For vesicles, it is a different
  story, because one major source of ill-conditioning is the bending
force.  Please quantify the effect of the preconditioner.}
\begin{itemize}
  \item We did a comparison of the number of GMRES iterations required
    with and without the block-diagonal preconditioner.  For the
    Taylor-Green flow, there is a 2X reduction in the number of
    iterations.

  \item In addition, because we are simulating rigid bodies, the
    block-diagonal preconditioner can be updated at each time step by
    simply applying a rotation matrix to the preconditioner from the
    previous time step.  This significantly reduces the cost of
    constructing the preconditioner and justifies its use.  In contrast,
    for deformable bodies such as vesicles, if an exact block-diagonal
    preconditioner is used, it must be constructed from scratch at every
    time step.
\end{itemize}

\noindent
\comment{Page 10, last line: ``semi-implicit time integrator proposed by
  Lu et al.~results in the STIV algorithm stalling." This should be
  elaborated more.  It is unclear why small minimal separation break the
  algorithm.  Is it due the semi-implicit scheme, the LCP solver, the
  linear solver?  Lu et al.~use spectral deferred correction for rigid
  bodies, is that effective here?}
\begin{itemize}
  \item The itemized environment in the start of Section 4 is meant to
    only offer the reader a brief summary of the examples to come.
    We slightly elaborated this section, but we reserve a more thorough
    discussion of the behavior for Section 4.2.

  \item In Section 4.2, we compare the contact algorithm with the two
    different time stepping schemes.  Since all other variables are held
    fixed (LCP solver, linear solver, etc.), we argue that it is the
    time stepping scheme that results in the stalling of the contact
    algorithm. 

  \item Regarding a higher-order time stepping method such as SDC, we
    expect that it is convergent only if the globally implicit method is
    used.  This prediction is based on a previous work of one author
    (Quaife) who developed the SDC framework for vesicle suspensions.
    While this is not complete the picture regarding SDC with small
    minimum separation distances, we leave more careful experiments and
    discussions to future work.

\end{itemize}

\noindent
\comment{Section 4.1: The conclusion of the section is unclear.  It is
  indeed expected that extra force in the system breaks
  time-reversibility.  The question is whether the method is convergent,
  i.e., using higher resolution (smaller h) it converges to the true
  solution and maintains reversibility.  Please clarify the significance
of this result.}
\begin{itemize}
  \item \todo[inline]{Bryan: Address this comment}
\end{itemize}

\noindent
\comment{Also, in Figure 3, what is drawn are not streamlines.  They are
Pathlines.}
\begin{itemize}
  \item The reviewer is correct.  The language in the caption of Figure
    3 and in the text that refers to Figure 3 has been modified.
\end{itemize}


\noindent
\comment{What is the limitation of the method with respect to the aspect
ratio $\lambda$?  What range of aspect ratio is feasible with this
method?  What causes the limitation.}
\begin{itemize}
  \item In theory the method works for any aspect ratio, but the cost is
    that additional resolution is required.  The resolution has to be
    sufficiently large to describe the geometry (curvature, arclength,
    etc.) and the hydrodynamic interactions between the rigid bodies and
    between the rigid bodies and the solid wall (ie.~resolve the density
    functions).
    
  \item We state in the {\em Limitations} section that larger aspect
    ratios our often present in applications, and such problems require
    higher resolutions than those we consider.

  \item In the {\em Taylor-Couette flow} section, we report the
    resolutions we used and state that higher resolutions are required
    to consider more slender bodies.

\end{itemize}

\noindent
\comment{Page 2, line 47, ``Deformable bodies, such as vesicles, deform as
they approach one another, and this creates a natural minimum separation
distance." Why is this true?  does this depend on flow/particle type?}
\begin{itemize}
  \item \todo[inline]{Bryan/Lukas: Find a reference.  Maybe Michael
    Graham?}
\end{itemize}

\noindent
\comment{Page 3, line 53, repulsion force in the context of contact
  algorithms typically refers to a force with a pre-defined potential.
  From what I understand for the work of Lu et al., they use a
  Lagrangian to impose the collision-free constraint and not a repulsion
  force.  Please rephrase.}
\begin{itemize}
  \item The reviewer is correct.  The work of Lu et al.~is a Lagrange
    multiplier method which acts as an {\em artificial repulsion force}
    (as described by Lu et al.).  We have modified the text to make it
    clear that the method is not a repulsion force in the traditional
    sense.
\end{itemize}

\noindent
\comment{Page 4, line 54, the symbol for torque is missing.}
\begin{itemize}
  \item This was a latex error that occurred in several locations---they
    have all been corrected.
\end{itemize}

\noindent
\comment{Page 5, line 25, ``To avoid contact, we will later relax the
  force- and torque-free conditions ...".  This is not correct.  In
  quasi-static Stokes flow, all freely suspended particles are always
  force-free, that is how the solution is computed.}
\begin{itemize}
  \item \todo[inline]{Bryan: See how Libin, Abtin, and Denis talked
    about this}
\end{itemize}

\noindent
\comment{Section 2.4 is well known and is better suited for the
appendix.}
\begin{itemize}
  \item We agree that this section is not necessary as it is described
    in many references.  We have removed the section entirely.  Instead,
    we added a small paragraph at the end of Section 2.2 that describes
    how the pressure and stress are simply new layer potentials, and the
    expressions are found in: \newline
    Power H. The completed double layer boundary integral equation
    method for two-dimensional Stokes flow.  IMA J Appl Math.
    1993;51(2):123--145.

  \item We briefly  describe how the pressure and stress are computed
    numerically in Section 3.5.  However, the method has already been
    described in previous work of one of the authors in: \newline
    Quaife B, Biros G. High-volume fraction simulations of
    two-dimensional vesicle suspensions. J Comput Phys.
    2014;274:245--267.
\end{itemize}


\noindent
\comment{Equation 9, is just the Stokes equation in energy form, which
is already given in Eq.~2.}
\begin{itemize}
  \item We agree that equations (9) and (2) are identical.  However, to
    address the constraint that the STIV is zero, one approach is to
    formulate the problem as a constrained optimization problem.  This
    is the approach taken by Liu et al.  Note, however, that we are only
    trying to highlight the relevant and key points of the work of Liu
    et al.
\end{itemize}

\noindent
\comment{Page 7, as a reference for ``Finally, using the divergence
theorem, the volumed average stresses can be expressed in terms of
boundary integrals" a book is used.  Please add page number/section that
clearly support the claim.}
\begin{itemize}
  \item This is described in Section 2.5 of Pozrikidis book: {\em Boundary
    integral and singularity methods}.
\end{itemize}

\noindent
\comment{Page 9, line 19, the force balance is addressed by increasing
the force on one particle.  Why is this not added as a constraint so
that the resulting force is balanced?}
\begin{itemize}
  \item The language in the previous version may have led to some
    confusion.  We do enforce a net force balance of 0.  The reviewer is
    right that this could be done by imposing this as a constraint.
    However, this would require an addition Lagrange multiplier and
    the result would be a modification to Equation (7).  
    
  \item We take a different, but simpler approach.  We simply modify one
    of the forces of a cluster of rigid bodies in near-contact so that
    it balances with the sum of the others.  By only modifying the
    smallest force, this makes sure that we are only modifying the least
    significant force.

  \item We have described both this techniques in the text.

\end{itemize}

\noindent
\comment{Section 3.3, the proposed adaptive time stepping scheme seems
  to be a heuristic.  What is the error bound for this method?  Given
  target accuracy at the final time, can it meet that error?  Is it
  effective in bringing down the cost of simulation for fixed target
accuracy?}
\begin{itemize}
  \item \todo[inline]{Bryan: It is a heuristic and we don't have error
    bounds.  This could be done, for example, with an embedded R-K
    method, a predictor-corrector, or SDC.  SDC is not implemented
    correctly (fix this)}
\end{itemize}



%%%%%%%%%%%%%%%%%%%%%%%%%%%%%%%%%%%%%%%%%%%%%%%%%%%%%%%%%%%%%%%%%%%%%%%%%%%%%%
\section*{Reviewer 3}
\noindent
\comment{While the paper gives a lot of details regarding various
  components of their solver, I feel that the paper is lacking in detail
  when it comes to discussing their main contribution, i.e.~how all
  interactions are handled implicitly. It is particularly confusing
  since the novelty of the approach is the fully-implicit handling of
  all hydrodynamic interactions, but the only time discretized equations
  in the manuscript are the update equations of the rigid body center of
  masses and orientations which are handled explicitly. It might be more
  instructive/clearer if the authors presented the complete discretized
  equations in section 3.3. This would also clarify the discussion of
the computational complexity of their approach in the subsequent
section.}
\begin{itemize}
  \item \todo[inline]{Bryan: Make it clearer even though we are doing
    forward Euler, it is important the bodies are coupled implicitly.
    Otherwise, the velocity $\mathbf{u}$ and $\omega$ are not
    sufficiently accurate and a smaller time step size must be taken.}
\end{itemize}

\noindent
\comment{Section 3.2, page 9, line 3, not only do bodies not intersect..
(missing ``not")}
\begin{itemize}
  \item Thank you for catching this typo.  The text is adjusted.
\end{itemize}

\noindent
\comment{The same symbol is used for describing self hydrodynamic
interactions and the hydrodynamic interactions between different
particle in equations 13,14 on Page 10.}
\begin{itemize}
  \item Thank you for pointing out this flaw in the notation.  We have
    updated equations (13) and (14) so that it is clearer how the
    integral equation is discretized.
\end{itemize}

\noindent
\comment{In equation 14, there is an inconsistent subscript as opposed
to a superscript for $\eta^{N+1}$}
\begin{itemize}
  \item Thank you for catching this typo.  The equation is corrected.
\end{itemize}



\end{document}
