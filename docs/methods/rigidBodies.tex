\documentclass[preprint, 10pt]{elsarticle}

\newcommand{\mcaption}[2]{\caption{\small \em #1}\label{#2}}
\newcommand{\secref}[1]{\ref{#1}}

\input{preambles.tex}

\begin{document}

\title{Methods paper for rigid bodies}

\author[Lukas]{Lukas Bystircky}
\author[Lukas]{Sachin Shanbhag}
\author[Bryan]{Bryan D.~Quaife}
\address[Lukas]{Department of Scientific Computing, Florida State University,
Tallahassee, FL, 32306.}
\address[Bryan]{Department of Scientific Computing and Geophysical Fluid
Dynamics Institute, Florida State University, Tallahassee, FL, 32306.}

\begin{abstract} 
We consider suspensions of rigid bodies in two dimensions \ldots
\end{abstract}

\begin{keyword}
  Stokes flow \sep Boundary integral method \sep Rigid body suspensions 
\end{keyword}

\maketitle





%%%%%%%%%%%%%%%%%%%%%%%%%%%%%%%%%%%%%%%%%%%%%%%%%%%%%%%%%%%%%%%%%%%%%%%
\section{Introduction\label{s:intro}}

\todo[inline]{Bryan will write this section}

This is a methods paper
\begin{itemize}
  \item Boundary integral equation formulation
  \item STIV
  \item FMM
  \item Near-singular integration
  \item Pressure and energy dissipation calculations
  \item Time integrator
\end{itemize}




%%%%%%%%%%%%%%%%%%%%%%%%%%%%%%%%%%%%%%%%%%%%%%%%%%%%%%%%%%%%%%%%%%%%%%%
\section{Formulation\label{s:formulation}} 
\subsection{Problem Formulation}

We consider a collection of rigid particles suspended in a
two-dimensional bounded or unbounded domain, $\Omega$, with boundary
$\partial\Omega$ (Figure~\ref{fig:geomSchematic}).  The boundary of the
fluid geometry is denoted by $\Gamma$ with $\Gamma_0$ being the
outermost boundary if the domain is bounded, and $\Gamma_i$, $1\leq i
\leq M_w$ are the other components of $\Gamma$.  The boundaries of rigid
particles are $\gamma_j$, $1\leq j\leq M_p$, and their union is denoted
$\gamma$.  Therefore, the boundary of the fluid domain is
\begin{align*}
  \partial\Omega =\Gamma \cup \gamma.
\end{align*} 
For each interior solid wall, we choose an interior point
$\cc^\Gamma_i$, and each rigid particle has an interior point
$\cc^\gamma_i$ and a corresponding orientation angle $\theta^\gamma_i$.
Finally, each solid wall has a net force and torque, $\FF^\Gamma_i$ and
$L^\Gamma_i$, respectively, that it applies to the fluid, and similar
forces and torques are defined for each rigid body.  A schematic of the
geometry is in Figure~\ref{fig:geomSchematic}.


%
%The fluid will be in a domain $\Omega$ with a boundary
%$\partial\Omega$. The boundary $\partial\Omega$ is the union of the
%surfaces of $N$ suspended particles each with a boundary $\gamma_k$,
%$1\leq k \leq N$, the surfaces of $M$ solid walls each with boundary,
%$\Gamma_\ell$, $1\leq\ell\leq M$ and optionally a containing wall
%denoted $\Gamma_0$. The suspended particles are all rigid and at each
%time step we will solve for their translational velocity
%$\mathbf{u}^{\tau}_k$ and angular velocity $\omega_k$, allowing us to
%update their centers and orientations, $\mathbf{c}_k$ and $\theta_k$
%respectively. Particles and interior walls will be undergoing a net
%force $\mathbf{F}_{k/\ell}$  and torque $L_{k/\ell}$. 

\begin{figure}[!h]
\begin{center}
\includegraphics{figures/multiply_connected.pdf}
\end{center}
\caption{\label{fig:geomSchematic}Sketch of a possible domain $\Omega$.
$\gamma_1$ and $\gamma_2$ enclose particles, while $\Gamma_1$ is a solid
wall. The outer boundary $\Gamma_0$ need not be present.  The vector
$\nn$ is the unit normal vector pointing into the fluid domain.}
\todo[inline]{Fix picture to make consistent with the text}
\todo[inline]{Shade in the rigid bodies like Libin did to classify rigid
bodies from holes in the geometry}
\end{figure}

%%%%%%%%%%%%%%%%%%%%%%%%%%%%%%%%%%%%%%%%%%%%%%%%%%%%%%%%%%%%%%%%%%%%%%%%%%%%%%%
\subsection{Governing Equations}\label{sec:governing}
We start by assuming that the fluid is governed by the incompressible
Navier-Stokes equations 
\begin{align*}
  \rho\left(\pderiv{\uu}{t} + (\uu \cdot \grad) \uu \right) &= 
    -\grad p + \mu \Delta \uu, \\
    \grad \cdot \uu &= 0,
\end{align*}
where $\uu$ is the velocity, $p$ is the pressure, $\rho$ is the fluid
density, and $\mu$ is the fluid viscosity.  The equations are
nondimensionalized by choosing a characteristic speed $U$ and
characteristic length scale $L$.  The resulting dimensionless number is
the Reynolds number $\Re = UL/\nu$ where $\nu$ is the kinematic fluid
viscosity.  We are interested in small particles and slow velocities
which renders the Reynolds number small $\Re \ll 1$.  Therefore, the
fluid is governed by the incompressible Stokes equations.  A Dirichlet
boundary condition $\UU$ is imposed on the solid walls $\Gamma$, and a
no-slip boundary condition is assumed for the rigid bodies $\gamma$.
Given the definitions in the previous section and the scales of
interest, the equations for the suspension of rigid bodies are given by
\begin{equation}
  \label{eqn:modelEquations}
  \begin{split}
  \mu \Delta \uu = \grad p, &\hspace{20pt} \xx \in \Omega, \gap
    &&\mbox{\em conservation of momentum,}\\
  \grad \cdot \uu = 0, &\hspace{20pt} \xx \in \Omega, \gap
    &&\mbox{\em conservation of mass,} \\
  \uu = \UU, &\hspace{20pt} \xx \in \Gamma, \gap 
    &&\mbox{\em wall velocity,} \\
  \uu = \uu^\tau_j + \omega_j(\xx-\cc^\gamma_j)^\perp,&\hspace{20pt} 
    \xx \in \gamma, \gap &&\mbox{\em no-slip on the particles,} \\
  \FF_j^\gamma = 0, &\hspace{20pt}j=1,\ldots,M_p, \gap 
    &&\mbox{\em force-free particles,} \\
  L_j^\gamma = 0, &\hspace{20pt}j=1,\ldots,M_p, \gap 
    &&\mbox{\em torque-free particles,}
  \end{split}
\end{equation}
Here, $\uu^\tau_j$ and $\omega_j$ are the translational and rotational
velocities of rigid body $j$, respectively, and $F_j^\gamma$ and
$L_j^\gamma$ are the net force and torque of rigid body $j$.  For the
suspension of rigid bodies, the fluid viscosity sets the time scale, and
we will assume it is one throughout the paper.  In the case of a bounded
domain, the boundary condition $\UU$ must satisfy the standard flux-free
condition, and in the case that the fluid domain is unbounded, the wall
velocity equation is replaced with a background flow condition
\begin{align*}
  \uu(\xx) = \uu_\infty(\xx), \quad |\xx| \rightarrow \infty.
\end{align*}

We will eventually relax the force-free and torque-free conditions.
Rather, a small force and torque will be imposed to guarantee that rigid
bodies do not come into unphysical contact.  This idea is first
described for vesicle suspensions by Lu et al.~\cite{Lu2017} and we
summarize the method in Section~\ref{sec:repulsion}.  Finally, given the
rigid body translational and rotational velocities, their centers and
inclination angles $(\cc_j,\theta_j)$, $j=1,\ldots,M_p$, satisfy
\begin{align*}
  \frac{d\cc_j}{dt} = \uu^\tau_j, \qquad 
  \frac{d\theta}{dt} = \omega_j.
\end{align*}

There exist many numerical methods for simulating the suspensions of
interfaces in a fluid by solving~\eqref{eqn:modelEquations}.  However,
given that the fluid equations are linear, we opt to use a boundary
integral equation (BIE) formulation~\cite{Pozrikidis1992}.  BIEs have
several advantages including that only the interface has to be tracked,
resulting in a dimension reduction and simplification of complex
geometries.  In the next section, we introduce the necessary integral
operators needed to reformulate~\eqref{eqn:modelEquations}.

%%%%%%%%%%%%%%%%%%%%%%%%%%%%%%%%%%%%%%%%%%%%%%%%%%%%%%%%%%%%%%%%%%%%%%%%%%%%%%%
\subsection{Boundary Integral Equation Representation}
Since the fluid equations are linear and homogeneous, a BIE formulation
is possible.  In addition to the advantageous dimension reduction, BIEs
achieve high-order accuracy, automatically satisfy the incompressibility
constraint, and can be solved with $\bigO(N)$ operations where $N$ is
the total number of points used to discretize $\bd\Omega$.  A more
detailed discussion of the advantages of a BIE formulation for the
incompressible Stokes equations can be found in Karrila and
Kim~\cite{Karrila1989}.

We start by formulating the incompressible Stokes equations in the
absence of rigid bodies.  The {\em double-layer potential} is the
convolution of the stresslet with an arbitrary density function.  In
particular, the double-layer potential is~\cite{Ladyzhenskaya1963,
Pozrikidis1992},
\begin{align}
  \label{eqn:dlp}
  \uu(\xx) = \DD[\eeta](\xx) = \frac{1}{\pi}\int_{\Gamma}
  \frac{\rr\cdot\nn}{\rho^2}\frac{\rr \otimes \rr}{\rho^2}
  \eeta(\yy)~\text{d}s_{\yy}, \quad \xx \in \Omega,
\end{align}
where $\rr = \xx - \yy$, $\rho=|\rr|$, and $\eeta$ is an unknown density
function defined on $\bd\Omega$.  The double-layer
potential~\eqref{eqn:dlp} satisfies the incompressible Stokes equations,
and the density function $\eeta$ satisfies
\begin{align*}
  \lim_{\substack{\xx \rightarrow \xx_0 \\ \xx \in \Omega}}
    \DD[\eeta](\xx) = \UU(\xx_0), \quad \xx_0 \in \Gamma,
\end{align*}
where $\UU$ is the imposed wall velocity.  When taking the limit, the
singularity of the stresslet results in a jump~\cite{Pozrikidis1992},
and the density function must satisfy  
\begin{align}
  -\frac{1}{2} \eeta(\xx_0) + \DD[\eeta](\xx_0) = \UU(\xx_0), 
    \quad \xx_0 \in \Gamma.
  \label{eqn:secondKindBIE}
\end{align}
The double-layer potential does satisfy the incompressible Stokes
equations, but it cannot represent all solutions of the incompressible
Stokes equations; in particular, it cannot represent rigid body motion.
This can be rectified by introducing point forces and torques due to
each interior component of the geometry $\Gamma_j$, $j =
1,\ldots,M_w$~\cite{Power1987, Power1993}.  This is done by introducing
the net force and torque about a point $\cc$ outside of $\Omega$
\begin{align*}
  \mathbf{S}(\xx,\cc) = -\log\rho\mathbf{I} + 
  \frac{\rr \otimes \rr}{\rho^2}, \quad \text{and} \quad
  \mathbf{R}(\xx,\cc) = \frac{\rr^\perp}{\rho^2},
\end{align*}
respectively, where $\rr = \xx - \cc$ and $\rho = |\rr|$.  Then, the
second-kind integral equation~\eqref{eqn:secondKindBIE} is replaced with
the completed second-kind BIE
\begin{equation}
  \label{eq:completed_DLP}
  \begin{split}
  -\frac{1}{2}\eeta(\xx_0) + \DD[\eeta](\xx_0) + 
    \sum_{j=1}^{M_w} \left(\mathbf{S}(\xx,\cc^\Gamma_j)\FF^\Gamma_j + 
      \mathbf{R}(\xx,\cc^\Gamma_j)L^\Gamma_j\right) &= \UU(\xx_0),
      \quad &&\xx_0 \in \Gamma, \\
  \int_{\Gamma_j} \eeta~\text{d}s &= \FF^\Gamma_j, 
      &&j=1,\ldots,M_w, \\
  \int_{\Gamma_j} \eeta\cdot (\xx - \cc^\Gamma_j)^\perp~\text{d}s &=   
      L^\Gamma_j, &&j=1,\ldots,M_w.
%                \sum_{j=1}^{M_p} \left(\mathbf{S}(\xx,\cc^\gamma_j)\FF^\gamma_j + 
%                \mathbf{R}(\xx,\cc^\gamma_j)L^\gamma_j\right)  =
  \end{split}
\end{equation}

We now introduce a suspension of rigid bodies $\gamma_j$,
$j=1,\ldots,M_p$.  Imposing the no-slip boundary condition on the rigid
bodies, the BIE formulation of the suspension of rigid bodies governed
by equation~\eqref{eqn:modelEquations} is
\begin{subequations}
  \label{eqn:BIEformulation}
  \begin{align}
    \UU(\xx) &= -\frac{1}{2}\eeta(\xx) + \DD[\eeta](\xx) +
    \sum_{j=1}^{M_w} \left(\mathbf{S}(\xx,\cc^\Gamma_j)\FF^\Gamma_j + 
      \mathbf{R}(\xx,\cc^\Gamma_j)L^\Gamma_j\right)  \nonumber \\
      &\hspace{96pt}+\sum_{j=1}^{M_p} \left(\mathbf{S}(\xx,\cc^\gamma_j)\FF^\gamma_j + 
      \mathbf{R}(\xx,\cc^\gamma_j)L^\gamma_j\right), 
    \quad \xx \in \Gamma, \label{eqn:BIEformulation1} \\
  \uu^\tau_j + \omega_j(\xx - \cc_j^\tau) &=
    -\frac{1}{2}\eeta(\xx) + \DD[\eeta](\xx) + 
    \sum_{j=1}^{M_w} \left(\mathbf{S}(\xx,\cc^\Gamma_j)\FF^\Gamma_j + 
      \mathbf{R}(\xx,\cc^\Gamma_j)L^\Gamma_j\right) \nonumber \\
    &\hspace{96pt}+\sum_{j=1}^{M_p} \left(\mathbf{S}(\xx,\cc^\gamma_j)\FF^\gamma_j + 
      \mathbf{R}(\xx,\cc^\gamma_j)L^\gamma_j\right), 
    \quad \xx \in \gamma, \label{eqn:BIEformulation2} \\
  \int_{\Gamma_j} \eeta~\text{d}s &= \FF^\Gamma_j, \quad
  \int_{\Gamma_j} \eeta\cdot (\xx - \cc^\Gamma_j)^\perp~\text{d}s =
  L^\Gamma_j, \quad j=1,\ldots,M_w, \label{eqn:BIEformulation3} \\
  \int_{\gamma_j} \eeta~\text{d}s &= \FF^\gamma_j, \quad
  \int_{\gamma_j} \eeta\cdot (\xx - \cc^\gamma_j)^\perp~\text{d}s =
  L^\gamma_j,\quad j=1,\ldots,M_p, \label{eqn:BIEformulation4} \\
  \FF^\gamma_j &= 0, \quad \L^\gamma_j = 0,\quad j=1,\ldots,M_p.
  \label{eqn:BIEformulation5}
\end{align}
\end{subequations}
This constitutes eight equations for eight unknowns: the density
function, net force, and net torque on the solid walls and rigid bodies,
and the translational and rotational velocities.  We have adopted the
method of Power~\cite{Power1993} to relate the density function on the
rigid bodies to the corresponding force and torque. \todo[inline]{I
don't understand this last sentence}

%We now choose the density functions, and the strengths of the Stokeslets
%and rotlets so that the boundary conditions
%\eqref{eq:boundary_condition} and \eqref{eq:particles_noslip} are
%satisfied.  The double layer potential has a jump discontinuity at the
%boundary of the geometry. On $\Gamma_0\cup\Gamma$, this leads to
%Fredholm equation,
%\begin{equation}
%  \label{eq:vel_walls} 
%  \begin{aligned}
% \uu_b(\xx) = -\frac{1}{2}\eeta(\xx) + \DD[\eeta](\xx) + 
%             \sum_{i=1}^{M_w} \left(\mathbf{S}(\xx,\cc^\Gamma_i)\FF^\Gamma_i + 
%                \mathbf{R}(\xx,\cc^\Gamma_i)L^\Gamma_i\right) + \sum_{j=1}^{M_p} \left(\mathbf{S}(\xx,\cc^\gamma_j)\FF^\gamma_j + 
%                \mathbf{R}(\xx,\cc^\gamma_j)L^\gamma_j\right),  
%  \end{aligned}
%\end{equation}
%while on $\gamma$ we get
%\begin{equation}  \label{eq:vel_particles} 
%\begin{aligned}
%  \uu^\tau_j + \omega_j(\xx-\cc^\gamma_j)^\perp &= -\frac{1}{2}\eeta(\xx) + \DD[\eeta](\xx) + \\&
%             \sum_{i=1}^{M_w} \left(\mathbf{S}(\xx,\cc^\Gamma_i)\FF^\Gamma_i + 
%                \mathbf{R}(\xx,\cc^\Gamma_i)L^\Gamma_i\right) + \sum_{j=1}^{M_p} \left(\mathbf{S}(\xx,\cc^\gamma_j)\FF^\gamma_j + 
%                \mathbf{R}(\xx,\cc^\gamma_j)L^\gamma_j\right).
% \end{aligned}
%\end{equation}         
While~\eqref{eqn:BIEformulation1} and~\eqref{eqn:BIEformulation2} are
both numerically desirable second-kind Fredholm integral equations
equations  for the unknown density function $\eeta(\xx)$,
equation~\eqref{eqn:BIEformulation1} still has a rank one null space
because of the flux-free condition of the boundary data
$\UU$~\cite{Ladyzhenskaya1963}.  Following~\cite{Power1993}, this null
space is removed by adding the term 
\begin{align}
\label{eq:N0_modification} 
  \mathcal{N}_0[\eeta](\xx) = \int_{\Gamma_0} 
    \nn(\xx)\otimes\nn(\yy)~\text{d}s(\yy)
\end{align}
to~\eqref{eqn:BIEformulation1}, but only for points $\xx \in \Gamma_0$.
Finally, if there are no solid walls, there is no null space, and
the only modification to~\eqref{eqn:BIEformulation} is that
equation~\eqref{eqn:BIEformulation1} is removed and
equation~\eqref{eqn:BIEformulation2} has the background velocity
$\uu_\infty(\xx)$ added to the right hand side.

%On $\Gamma$ we have prescribed the velocity $\uu_b(\xx)$,
%but we need to solve for the net force and torque on each wall. On
%$\gamma$ on the other hand we know the net force and torque beforehand
%and must solve for the translational and rotational velocity of each
%particle. 
%
%A simple counting argument shows that this system is indeterminate since we have more unknowns than equations. We follow Power~\cite{Power1993} and
%relate the density function $\eeta$ on each solid wall $\Gamma_i$ and
%each rigid body $\gamma_i$ to their corresponding force and torque
%\begin{equation}
%  \label{eq:closure}
%  \begin{aligned}
%    \int_{\Gamma_i} \eeta~\text{d}s &= \FF^\Gamma_i, \quad
%    \int_{\Gamma_i} \eeta\cdot (\xx - \cc^\Gamma_i)^\perp~\text{d}s = L^\Gamma_i, \\
%    \int_{\gamma_j} \eeta~\text{d}s &= \FF^\gamma_j, \quad
%    \int_{\gamma_j} \eeta\cdot (\xx - \cc^\gamma_j)^\perp~\text{d}s = L^\gamma_j.
%  \end{aligned}
%\end{equation}



%Combining \eqref{eq:vel_walls}, \eqref{eq:vel_particles} and \eqref{eq:closure}we can write our problem in the compact notation,
%\begin{equation}\label{eq:stokes_unbounded} \begin{bmatrix} -\frac{1}{2} +
%\mathcal{D} & 1 & (\mathbf{x}-\mathbf{c})^\perp & \mathcal{S} & \mathcal{R}\\
%		\int \cdot~ \text{d}s & 0 & 0 & 0 & 0\\
%		\int\cdot(\mathbf{x}-\mathbf{c})^\perp~\text{d}s & 0 & 0 & 0 & 0\\
%		\int \cdot~ \text{d}s & 0 & 0 & - 1 & 0\\
% \int\cdot(\mathbf{x}-\mathbf{c})^\perp~\text{d}s & 0 & 0 & 0 & -1\end{bmatrix}%\begin{bmatrix}
%	\pmb{\eta}\\\mathbf{u}^\tau \\ \pmb{\omega} \\ \mathbf{F}_w \\\mathbf{ L}_w
%\end{bmatrix}
%=
%\begin{bmatrix}
%	-\mathbf{u}_{\infty} - \mathcal{S}\mathbf{F}_p - \mathcal{R}\mathbf{L}_p\\
%	\mathbf{F}_p\\
%	\mathbf{L}_p\\
%	0\\
%	0
%\end{bmatrix}
%\end{equation}

%%%%%%%%%%%%%%%%%%%%%%%%%%%%%%%%%%%%%%%%%%%%%%%%%%%%%%%%%%%%%%%%%%%%%%%%%%%%%%%%%\subsubsection{Bounded Domains}
%
%Bounded domains lead to a similar system, however it must be modified
%slightly. For fluid inside a container the double layer potential has a
%rank one null space~\cite{Ladyzhenskaya1963}. Following~\cite{Power1993}
%this null space can be removed by adding an operator that is active only
%over the enclosing boundary $\Gamma_0$,
%\[ \mathcal{N}_0[\pmb{\eta}](\mathbf{x}) = \delta_{i0} \int_{\Gamma_i}\mathbf{n}(\mathbf{x})\otimes\mathbf{n}(\mathbf{y})~\text{d}s(\mathbf{y}).\]
%The compatibility condition~\eqref{eq:compatibility} ensures that this term evaluates to 0. Adding $\mathcal{N}_0$ to \eqref{eq:stokes_unbounded} and removing the background flow leads to the linear system,
%\begin{equation}\label{eq:stokes_bounded} \begin{bmatrix} -\frac{1}{2} + \mathcal{D} + \mathcal{N}_0 & 1 & (\mathbf{x}-\mathbf{c})^\perp & \mathcal{S} & \mathcal{R}\\
%		\int \cdot~ \text{d}s & 0 & 0 & 0 & 0\\
%		\int\cdot(\mathbf{x}-\mathbf{c})^\perp~\text{d}s & 0 & 0 & 0 & 0\\
%		\int \cdot~ \text{d}s & 0 & 0 & - 1 & 0\\
% \int\cdot(\mathbf{x}-\mathbf{c})^\perp~\text{d}s & 0 & 0 & 0 & -1\end{bmatrix}%\begin{bmatrix}
%	\pmb{\eta}\\\mathbf{u}^\tau \\ \pmb{\omega} \\ \mathbf{F}_w \\\mathbf{ L}_w
%\end{bmatrix}
%=
%\begin{bmatrix}
%	 - \mathcal{S}\mathbf{F}_p - \mathcal{R}\mathbf{L}_p\\
%	\mathbf{F}_p\\
%	\mathbf{L}_p\\
%	0\\
%	0
%\end{bmatrix}
%\end{equation}



%%%%%%%%%%%%%%%%%%%%%%%%%%%%%%%%%%%%%%%%%%%%%%%%%%%%%%%%%%%%%%%%%%%%%%%%%%%%%%%
\subsection{Repulsion Forces}
\label{sec:repulsion}
\todo[inline]{Tone back this section significantly  Can we show that the
net force and torque are always zero?  This is important otherwise the
solution becomes unbounded as per Stokes paradox.  Libin's old code
didn't support non-circular rigid particles}

\begin{figure}[!h]\label{fig:collision_sketch}
\begin{center}
\includegraphics{figures/collisions.pdf}
\end{center}
\caption{Sketch of potential collisions.}
\end{figure}
The exact solutions of the Stokes equations prohibit contact between
force-free and torque-free particles in finite time due to lubrication
forces. When evaluating a target point close to the boundary (as is necessary when two particles come close together) the kernel in the double layer potential becomes very sharply peaked and thus difficult to integrate accurately. Spatial and temporal adaptivity \cite{Kropinski1999}, special integration techniques \cite{Klockner2013, Ying2006} or asymptotic expansions of lubrication forces \cite{Mammoli2006} are all tools that can help, however even if we evaluate the double layer potential accurately, time stepping errors can still lead to contact. To keep computational costs reasonable we must turn to alternative approaches. 

One such approach is to introduce an artificial repulsion force. There
are many possible choices for the type of force. One possibility is a repulsion force based on a Morse or 
Lennard -Jones type potential that grows as a high order polynomial as two particles become close together \cite{Flormann2017, Liu2006}. This has been shown to work for dense
suspensions, however the resulting ODEs become very stiff as the
separation between particles decreases, thus requiring smaller time
steps. Spring based models \cite{Tsubota2006, Zhao2013, Kabacogulu2017} have also been used to generate artificial repulsion forces.

Our approach mirrors \cite{Lu2017} whereby we choose the forces in such a way as to
explicitly guarantee each time step is collision free.  At each time
step $t^n$ the Stokes equations are solved and the particles are
advanced to a candidate configuration at $t^{n+1}$.  Using a linear
interpolant, we check if any collisions occurred in the interval
$[t^n,t^{n+1}]$.  If the time step is contact-free, the candidate
configuration is accepted.  If contact is detected, the candidate
solution is rejected and we resolve
equations~\eqref{eq:vel_walls},~\eqref{eq:vel_particles},
and~\eqref{eq:closure}, but with artificial repulsion forces $\FF^\gamma_j$ and torques
$L^\gamma_j$ that are chosen to try and avoid contact.  This lets us form a new candidate configuration. This process is repeated until we end up with a contact-free configuration.


%%%%%%%%%%%%%%%%%%%%%%%%%%%%%%%%%%%%%%%%%%%%%%%%%%%%%%%%%%%%%%%%%%%%%%%%%%%%%%%
\subsection{Avoiding Contact with STIV}
Before discussing how collisions can be resolved, we first define a
metric that measures collision.  This metric should track all pairwise
collisions and detect if a two particles overlapped not only at the
discrete time points, but at any time.  We let $\mathbf{V}(t)$ be a
vector with size $\binom{M_p}{2}$ which is the total number of possible
pairwise collisions.  $\mathbf{V}$ should be defined in such a way that
it is 0 if no collisions have occurred, and if there is a collision, its
value should quantify the amount of overlap.   Then, if there is a
collision between two particles, the repulsion force should be chosen to
scale with the magnitude of the corresponding entry of $\mathbf{V}$.
There are several possible choices for $\mathbf{V}(t)$, the simplest
being a signed distance between all points on all particles. We use the
concept of {\em Space-Time Interference Volumes} (STIV) introduced by
Harmon et al.~\cite{Harmon2011} and adapted for the suspension of
deformable and rigid particles~\cite{Lu2017}. STIVs involve a time
integral and are therefore more expensive to compute than other metrics
(e.g. a signed distance). In most simulations however the cost of
computing an STIV is dwarfed by the cost of the linear solves necessary
to compute the velocities of the particles. The advantage of STIVs
compared to simpler metrics is that under the assumption of ballistic
motion they are able to detect collisions between two particles that
pass completely through one another, i.e. it can detect that a collision
has occurred even if the final candidate configuration is contact-free.





%%%%%%%%%%%%%%%%%%%%%%%%%%%%%%%%%%%%%%%%%%%%%%%%%%%%%%%%%%%%%%%%%%%%%%%%%%%%%%%
\subsection{Variational Formulation}

The incompressible Stokes equations can be can be restated as a minimization
problem. Consider the functional,
\[ \mathcal{J}(\mathbf{u}) = \int_{\Omega} \nabla\mathbf{u}:\nabla\mathbf{u} -
2\mathbf{f}\cdot\mathbf{u} ~\text{d}\Omega,\]
and the associated constrained minimization problem,
\[ \min \mathcal{J}(\mathbf{u}) ~:~ \nabla\cdot\mathbf{u} = 0 \text{ in
}\Omega.\]
Introducing $p$, a Lagrange multiplier for the incompressibility condition, we
can construct a Lagrangian for this system,
\begin{equation}\label{eq:lagrangian} \mathcal{L}(\mathbf{u},p) =
\mathcal{J}(\mathbf{u}) - \int_{\Omega}
2p\nabla\cdot\mathbf{u}~\text{d}\Omega.\end{equation}
First order optimality (KKT) conditions for $\mathcal{L}(\mathbf{u},p)$
recover the incompressible Stokes equations. For our problem, in
addition to the incompressibility condition, we wish to enforce the
constraint that the solution $\mathbf{u}$ at a time $t_0$ should not
introduce collisions at time $t_0+\Delta t$, in other words
$\mathbf{V}(t_0 + \Delta t) \geq \mathbf{0}$.  This constraint can be
incorporated in the Lagrangian \eqref{eq:lagrangian} with the
introduction of a Lagrange multiplier $\pmb{\lambda}$ with one component
foreach possible collision volume,
\begin{equation}\label{eq:lagrangian2} \tilde{\mathcal{L}}(\mathbf{u},p,\lambda)= \mathcal{L}(\mathbf{u},p) + \pmb{\lambda} \cdot \mathbf{V}(t_0+\Delta
t).\end{equation}
First order optimality for \eqref{eq:lagrangian2} yields the Stokes equations
with a modified forcing function,
\begin{equation}\label{eq:stokes_mod}-\Delta \mathbf{u} + \nabla p = \mathbf{f}
+ \int_{\Omega} \text{d}_{\mathbf{u}} \mathbf{V}^T\pmb{\lambda}
~\text{d}\Omega,\end{equation}
subject to the constraints
\[ \nabla\cdot\mathbf{u} =0, ~\mathbf{V}(t_0 + \Delta t) \geq 0,~\pmb{\lambda}
\geq 0, ~ \pmb{\lambda}\cdot\mathbf{V}(t_0+\Delta t) = 0. \]

The constraints on $\mathbf{V}$ and $\pmb{\lambda}$ can be combined into a
single constraint,
\begin{equation}\label{eq:ncp_constraint} \mathbf{V}(t_0 + \Delta t)\geq
\mathbf{0} \perp \pmb{\lambda}\geq \mathbf{0}.\end{equation}


%%%%%%%%%%%%%%%%%%%%%%%%%%%%%%%%%%%%%%%%%%%%%%%%%%%%%%%%%%%%%%%%%%%%%%%%%%%%%%%
\subsection{Incorporating Repulsion Forces}

The addition of a forcing term to the right hand side of the Stokes equations
would normally lead to a volume integral. However, in this case since
$\text{d}_{\mathbf{u}} V$ can be non-zero only on the boundary, we can capture
the repulsion force by adding a net force and torque to each particle or wall as needed. The net force $\mathbf{F}^k_p$ and torque $L^k_p$ are given by,
\[ \mathbf{F}^k_p = \int_{\Gamma_k}
\text{d}_\mathbf{u}V^T\pmb{\lambda}~\text{d}s, \qquad L_p^k = \int_{\Gamma_k}
\text{d}_\mathbf{u}V^T\pmb{\lambda}\cdot(\mathbf{x}-\mathbf{c}_k)^\perp~\text{d}s.\]

%%%%%%%%%%%%%%%%%%%%%%%%%%%%%%%%%%%%%%%%%%%%%%%%%%%%%%%%%%%%%%%%%%%%%%%%%%%%%%%
\subsection{Complementary Problem}

To compute the repulsion forces we must first compute $\pmb{\lambda}$ for each
time step such that \eqref{eq:ncp_constraint} is satisfied.
Consider particles suspended in a fluid in an ambient flow $\mathbf{u}_\infty$.
This flow can be an imposed background flow or come from solid walls. In the
later case this velocity field is computed by solving the appropriate resistance problem. In either case this flow can be expressed as the sum of a translational component $\mathbf{u}_{\infty}^\tau$, a rotational component $\omega_{\infty}$ and a strain component $\mathbf{e}_{\infty}$,
\[ \mathbf{u}_{\infty}(\mathbf{x}) = \mathbf{u}_{\infty}^\tau +
\omega_\infty\times \mathbf{x} + \mathbf{e}_\infty \cdot\mathbf{x}.\]
The translational and rotational velocity as well as the density
function of a collection of particles can be computed from the force,
torque and strain rate~\cite{Karrila1991},
\begin{equation}\label{eq:mobility} \begin{bmatrix} \mathbf{u}_\infty^\tau -
\mathbf{u}_\tau\\ \omega^\infty - \omega \\\pmb{\eta}\end{bmatrix} =
\mathcal{M}\begin{bmatrix}\mathbf{F}\\\mathbf{L}\\\mathbf{e}_\infty\end{bmatrix},\end{equation}
where $\mathcal{M}$ is the {\em mobility tensor} and depends only on the
particle configuration $\mathbf{q}^0$ at some time $t^0$. Assuming the
only force and torques acting on particles arises from repulsion forces,
we can decompose \eqref{eq:mobility} as,
\[ \begin{bmatrix} \mathbf{u}_\infty^\tau - \mathbf{u}_\tau\\ \omega^\infty -
\omega \\\pmb{\eta}\end{bmatrix} =
\mathcal{M}\begin{bmatrix}\mathbf{0}\\\mathbf{0}\\\mathbf{e}_\infty\end{bmatrix}+
\mathcal{M}\begin{bmatrix}\mathbf{F}_c\\\mathbf{L}_c\\\mathbf{0}\end{bmatrix}.\]
Once we solve for $\mathbf{u}_\tau$ and $\omega$ we can update the positions an dangles of each particle using an explicit Euler step. With an abuse of notation,
this lets us express a candidate configuration $\mathbf{q}^{1}$ as,
\[ \mathbf{q}^{1} = \mathbf{q}^n + \Delta
t\left(\mathcal{M}\begin{bmatrix}\mathbf{0}\\\mathbf{0}\\\mathbf{e}_\infty\end{bmatrix}
+
\mathcal{M}\begin{bmatrix}\mathbf{F}_c\\\mathbf{L}_c\\\mathbf{0}\end{bmatrix}\right).\]

This candidate configuration must satisfy the constraint
\eqref{eq:ncp_constraint}, which we will rewrite to show the dependence of
$\mathbf{V}$ on both $\mathbf{q}^0$ and $\mathbf{q}^{1}$,
\begin{equation}\label{eq:ncp_new}\mathbf{V}(\mathbf{q}^0,\mathbf{q}^{1})\geq
\mathbf{0}\perp \pmb{\lambda}^n\geq \mathbf{0} \Rightarrow
\mathbf{V}\left(\mathbf{q}^0, \mathbf{q}^0 + \Delta
t\left(\mathcal{M}\begin{bmatrix}\mathbf{0}\\\mathbf{0}\\\mathbf{e}_\infty\end{bmatrix}
+
\mathcal{M}\begin{bmatrix}\mathbf{F}_c\\\mathbf{L}_c\\\mathbf{0}\end{bmatrix}\right)\right)
\geq \mathbf{0} \perp\pmb{\lambda}\geq \mathbf{0} .\end{equation}

This is a nonlinear complementary problem (NCP). We can see this by explicitly
including the dependence of $\mathbf{V}$ on $\pmb{\lambda}$,
\[\mathbf{V}\left(\mathbf{q}^0, \mathbf{q}^0 + \Delta
t\left(\mathcal{M}\begin{bmatrix}\mathbf{0}\\\mathbf{0}\\\mathbf{e}_\infty\end{bmatrix}
+ \mathcal{M}\begin{bmatrix} \int_{\Gamma_k}
\text{d}_\mathbf{u}\mathbf{V}^T\pmb{\lambda}~\text{d}s\\ \int_{\Gamma_k}
\text{d}_\mathbf{u}\mathbf{V}^T\pmb{\lambda}\cdot(\mathbf{x}-\mathbf{c}_k)^\perp~\text{d}s
\\\mathbf{0}\end{bmatrix}\right)\right) \geq \mathbf{0} \perp\pmb{\lambda}\geq
\mathbf{0} .\]

A first order linearization of this NCP turns it into a sequence of linear
complementary problems (LCP). Starting from an initial guess for
$\pmb{\lambda}$, $\pmb{\lambda}^0$ the following sequence should converge to the solution of \eqref{eq:ncp_new}:
\begin{equation}\label{eq:lcp}\begin{aligned}
\mathbf{V}\biggl(\mathbf{q}^0 + \Delta
t\biggl(\mathcal{M}\begin{bmatrix}\mathbf{0}\\\mathbf{0}\\\mathbf{e}_\infty\end{bmatrix}
&+ \mathcal{M}\begin{bmatrix} \int_{\Gamma_k}
\text{d}_\mathbf{u}\mathbf{V}^T\pmb{\lambda}^\ell~\text{d}s\\ \int_{\Gamma_k}
\text{d}_\mathbf{u}\mathbf{V}^T\pmb{\lambda}^\ell\cdot(\mathbf{x}-\mathbf{c}_k)^\perp~\text{d}s
\\\mathbf{0}\end{bmatrix}\biggr)\biggr) \\
&+ \Delta t \mathcal{M}\begin{bmatrix}\int_{\Gamma_k}
\text{d}_\mathbf{u}\mathbf{V}^T\pmb{\lambda}^{\ell+1}~\text{d}s\\
\int_{\Gamma_k}
\text{d}_\mathbf{u}\mathbf{V}^T\pmb{\lambda}^{\ell+1}\cdot(\mathbf{x}-\mathbf{c}_k)^\perp~\text{d}s
\\\mathbf{0}\end{bmatrix}\frac{\partial\mathbf{V}}{\partial \mathbf{q}^1} \geq
\mathbf{0} \perp \pmb{\lambda}^{\ell+1} \geq
\mathbf{0}.\end{aligned}\end{equation}

The sequence \eqref{eq:lcp} will be solved at each time step. 

%%%%%%%%%%%%%%%%%%%%%%%%%%%%%%%%%%%%%%%%%%%%%%%%%%%%%%%%%%%%%%%%%%%%%%%
\subsection{Computing Pressures and Stresses}
To understand the rheological and statistical properties of suspensions
of rigid bodies, it is necessary to compute quantities such as the
pressure, stress, and energy dissipation.  In the BIE setting, computing
these quantities is entirely a post-processing step, and they also be
computed with high-order accuracy using methods described in
Section~\ref{s:method}.

To compute the pressure, we start with the conservation of momentum
\begin{align*}
  \nabla p = \Delta \uu,
\end{align*}
and substitute the double-layer potential~\eqref{eqn:dlp}.  By
integrating, the pressure of the double-layer potential is
\begin{align*}
  p(\xx) = \frac{1}{\pi}\int_{\partial\Omega}\frac{1}{\rho^2}\left((
    \mathbf{I} - 2(\rr \otimes \rr)\nn\right)\cdot\pmb{\eta}~\text{d}s. 
\end{align*}
The pressure due to the Stokeslets and rotlets can be easily computed,
and the pressure resulting from the completed double-layer potential
formulation for the velocity~\eqref{eqn:completed_DLP} is
\begin{align}
  \label{eqn:pressure} 
  p(\xx) = \frac{1}{\pi}\int_{\partial\Omega}\frac{1}{\rho^2}\left((
    \mathbf{I} - 2(\rr \otimes \rr)\nn\right)\cdot\pmb{\eta}~\text{d}s +
  \sum\limits_{i=1}^{M_w}
  \frac{\FF_i^\Gamma\cdot(\xx-\cc_i^\Gamma)}{2\pi||\xx-\cc_i^{\Gamma}||^2}
  + \sum\limits_{i=1}^{M_p}
    \frac{\FF_i^\gamma\cdot(\xx-\cc_i^\gamma)}{2\pi|\xx-\cc_i^\gamma|^2}.
\end{align}

To compute the stress tensor
\begin{align*} 
  \pmb{\sigma} = -p \mathbf{I} + \left(\nabla \uu + (\nabla\uu)^T\right),
\end{align*}
we start by using the layer potential representation of the
pressure~\eqref{eqn:pressure}.  Then, the viscous stress tensor is
computed by computing the Jacobian of the double-layer
potential~\eqref{eqn:dlp} and the stokeslets and Rotlets to form
\begin{align}
  \label{eq:stress}
  \ssigma = \ssigma_{\eeta} + \ssigma_{S} + \ssigma_{R},
\end{align}
where
\begin{align*}
  \ssigma_{\eeta}(\xx) &= \frac{1}{\pi}\int_{\partial\Omega}\left( 
    \frac{\nn\cdot\eeta}{\rho^2}\mathbf{I} 
    -\frac{(\rr\cdot\nn)(\rr\cdot\eeta)(\rr\otimes\rr)}{\rho^6} 
    +\frac{(\rr\cdot\nn)(\rr\otimes\eeta + \eeta\otimes\rr)}{\rho^4} 
    + \frac{(\rr\cdot\eeta)(\rr\otimes\nn + \nn\otimes\rr)}{\rho^4}\right)\text{d}s,\\
  \ssigma_S(\xx) &= \sum\limits_{i=1}^{M_w} 
    \frac{\FF_i^\Gamma\cdot(\cc_i^\Gamma-\xx)}{\pi|\xx-\cc_i^\Gamma|^2}
        (\cc_i^\Gamma - \xx)\otimes(\cc_i^\Gamma-\xx)  + 
    \sum\limits_{i=1}^{M_p}
    \frac{\FF_i^\gamma\cdot(\cc_i^\gamma-\xx)}{\pi|\xx-\cc_i^\gamma|^2}
        (\cc_i^\gamma - \xx)\otimes(\cc_i^\gamma-\xx),\\
  \ssigma_R(\xx) &= \sum\limits_{i=1}^{M_w} \frac{L_i^\Gamma}{2\pi|\xx-\cc_i^\Gamma|^2}
        ((\cc_i^\gamma - \xx)\otimes(\cc_i^\gamma-\xx)^\perp + 
    (\cc_i^\gamma - \xx)^\perp\otimes(\cc_i^\gamma-\xx))  \\
  &\qquad\qquad\qquad + \sum\limits_{i=1}^{M_p} \frac{L_i^\gamma}{2\pi|\xx-\cc_i^\gamma|^2}
    ((\cc_i^\gamma - \xx)\otimes(\cc_i^\gamma-\xx)^\perp + 
    (\cc_i^\gamma - \xx)^\perp\otimes(\cc_i^\gamma-\xx)).
\end{align*}
		
%%%%%%%%%%%%%%%%%%%%%%%%%%%%%%%%%%%%%%%%%%%%%%%%%%%%%%%%%%%%%%%%%%%%%%%
\section{Numerical Methods\label{s:method}} 
\todo[inline]{Need a lot more on the time stepping method and how ours
is novel}
We use a Lagrangian formulation where we track the centers $\cc_i$ and
orientations $\theta_i$ of each rigid body $\gamma_i$.  We use a fully
implicit time stepping method and a spectral discretization in space.  All
the integral equations are discretized with a Nystr\"om  methods, and
interactions between nearly touching interfaces is resolved with a
near-singular integration scheme.  Finally, the Fast Multipole Method
(FMM) is used to accelerate the matrix-vector multiplications that arise
upon discretization.  By using these methods, spectral accuracy can be
achieved for long time horizons with optimal complexity.

%%%%%%%%%%%%%%%%%%%%%%%%%%%%%%%%%%%%%%%%%%%%%%%%%%%%%%%%%%%%%%%%%%%%%%%
\subsection{Spatial Discretization}
Letting $\xx(\alpha)$ be a parameterization of a rigid body or a solid
wall, we can represent any smooth function defined on this curve using a
Fourier series as
\begin{align}
  f(\alpha) = f(\xx(\alpha)) = \sum_{k \in \ZZ} \hat{f}_k e^{ik\alpha}.
\end{align}
The FFT is used to compute $\hat{f}$, and all derivatives are computed
with this Fourier series so that spectral accuracy is achieved.

The double-layer potentials in~\eqref{eqn:BIEformulation} are
discretized with a Nystr\"om method.  Because the kernels are smooth,
the trapezoid rule guarantees spectral accuracy~\cite{tre-wei2014}.  For
example, equation~\eqref{eqn:BIEformulation1} is discretized as
\begin{equation*}
  \begin{aligned}
  \UU(\xx_i) = -\frac{1}{2}\eeta(\xx_i) + 
  \sum_{k=1}^{N} K(\xx_i,\xx_k) \eeta(\xx_k) \Delta s_k
    &+ \sum_{j=1}^{M_w} \left(\mathbf{S}(\xx_i,\cc^\gamma_j)\FF_j +
    \mathbf{R}(\xx_i,\dd_j)L_j\right)  \\
    &+ \sum_{j=1}^{N} \left(\mathbf{S}(\xx_i,\cc^\Gamma_j)\FF_j +
    \mathbf{R}(\xx_i,\cc^\Gamma_j)\L_j\right),
  \end{aligned}
\end{equation*}
where $N = M_w N_w + M_p N_p$ is the total number of discretization
points and
\begin{align*}
  K(\xx,\yy) = \frac{1}{\pi} \frac{\rr \cdot \nn}{\rho^2} 
               \frac{\rr \otimes \rr}{\rho^2}
\end{align*}
is the kernel of the double-layer potential.
Equation~\eqref{eqn:BIEformulation2} is discretized in a similar
fashion.  For the diagonal entries, we use the limiting value of $K$
\begin{align*}
  \lim_{\substack{\yy \rightarrow \xx \\ \yy \in \bd\Omega}} 
    K(\xx,\yy) = \frac{\kappa}{2\pi}\tt\otimes\tt,
    \quad \xx \in \bd\Omega
\end{align*}
where $\kappa(\xx)$ is the curvature and $\tt(\xx)$ is the tangent
vector of $\bd\Omega$ at $\xx$.

The trapezoid rule achieves spectral accuracy when both the source
points $\yy$ and the target point $xx$ are on the same solid wall
$\Gamma_k$ or rigid particle $\gamma_k$.  However, when the target point
is on a different body, the accuracy of the trapezoid rule can
deteriorate which leads to instabilities.  This increase in error
happens when the target point is close to the source points, but not on
the same body as the source points.  To resolve this issue, an algorithm
for near-singular integration method must be employed.  There are now
several methods for near-singular integration including
regularizations~\cite{bea-yin-wil2016, bea-lai2001}, quadrature by
expansion~\cite{Klockner2013}, barycentric
interpolation~\cite{bar-wu-vee2015}, and panel-based
quadrature~\cite{hel-oja2008}.  In this work, we use an interpolation
based scheme~\cite{Ying2006} which has been used for other
two-dimensional suspensions~\cite{Quaife2014}.

Once each of the
equations~\eqref{eq:vel_walls},~\eqref{eq:vel_particles},
and~\eqref{eq:closure} have been discretized using the above described
quadrature rules, the result is a dense $N \times N$ linear system.
Since this linear system is the discretization of a second-kind integral
equation, GMRES~\cite{Saad1986} converges in a mesh-independent number of
iterations~\cite{cam-ips-kel-mey-xue1996}.  Since GMRES only requires
matrix-vector multiplications, the algorithmic cost per time step is
proportional to the cost of doing a single matrix-vector multiply.


%%%%%%%%%%%%%%%%%%%%%%%%%%%%%%%%%%%%%%%%%%%%%%%%%%%%%%%%%%%%%%%%%%%%%%%
\subsection{Time Stepping Methods}
\todo[inline]{Bryan can write this section}

Once we solve for the translational and angular velocity of the
particles, the position and angle of each particle are updated according
to the ODEs,
\[ \frac{\text{d}}{\text{d}t}\mathbf{c}_k = \mathbf{u}^\tau_k, \qquad
\frac{\text{d}}{\text{d}t}\theta_k =\omega_k.\]
The ODEs are advanced in time using an explicit Euler step. 


%%%%%%%%%%%%%%%%%%%%%%%%%%%%%%%%%%%%%%%%%%%%%%%%%%%%%%%%%%%%%%%%%%%%%%%
\subsection{Fast Summation and Preconditioning}
The linear system arising from ~\eqref{eq:vel_walls},~\eqref{eq:vel_particles},
and~\eqref{eq:closure}  is discretized using a collocation trapezoid
method. For particles that are in near-contact the near singular
integration scheme described in~\cite{Quaife2014, Ying2006} is used. The
discretized system is solved with GRMES~\cite{Saad1986} with a block
diagonal preconditioner and accelerated using the fast multipole method
\cite{Greengard1987}.  Since the linear system arises from a second kind
Fredholm equation the condition number of the matrix is bounded and does
not increase with finer resolution. The number of GMRES iterations is
therefore mesh resolution independent. This leads to a solver that is
$O(n)$, where $n$ is the number of mesh points. 

The matrices described in \eqref{eq:vel_walls} and
\eqref{eq:vel_particles} are full. They can be made block-diagonal by
treating inter-particle interactions explicitly and moving them to the
right hand side. This is termed {\em locally implicit} and is described
in \cite{Lu2017}. For dense suspensions however this type of time
stepping can lead to instabilities as particles become tightly packed. 

\begin{comment}
\begin{algorithm}
	 \SetKwInOut{Input}{Input}
    	\SetKwInOut{Output}{Output}

  	  \underline{Collision free time stepper}\;
\Input{collision free configuration $\mathbf{q}^0$, time step size $\Delta t$}
\Output{collision free configuration $\mathbf{q}^1$ }
	 $\mathbf{u}^* \gets \mathbf{A}(\mathbf{q}^0,\mathbf{0},\mathbf{0})$\;
	$\mathbf{q}^* \gets \mathbf{q}^0 + \Delta t \mathbf{u}^*$\;
	Compute $\mathbf{V}(\mathbf{q}^0,\mathbf{q}^*)$, $\text{d}_u\mathbf{V}$\;
	\While {$\mathbf{V} < \mathbf{0}$}
	{
		$\pmb{\lambda} \gets$ LCP\_solve($\text{d}_{\mathbf{u}}\mathbf{V}$)\;
$\mathbf{F}_k \gets \int_{\Gamma_k}
\text{d}_{\mathbf{u}}\mathbf{V}^T\pmb{\lambda}~\text{d}s$\;
$L_k \gets \int_{\Gamma_k}
\text{d}_{\mathbf{u}}\mathbf{V}^T\pmb{\lambda}\cdot(\mathbf{x}-\mathbf{c}_k)^\perp~\text{d}s$\;
		$\mathbf{u}^* \gets \mathbf{A}(\mathbf{q}^0,\mathbf{F}_k,\mathbf{L}_k)$\;
		$\mathbf{q}^* \gets \mathbf{q}^0 + \Delta t \mathbf{u}^*$\;
		Compute $\mathbf{V}(\mathbf{q}^0,\mathbf{q}^*)$, $\text{d}_u\mathbf{V}$\;
	}
\end{algorithm}
\end{comment}

%%%%%%%%%%%%%%%%%%%%%%%%%%%%%%%%%%%%%%%%%%%%%%%%%%%%%%%%%%%%%%%%%%%%%%%
\section{Results\label{s:results}} 

\subsection{Shear Flow}
\todo[inline]{Plot stream lines by doing many runs and showing that they
cross when contact first happens}

A simple numerical experiment done in \cite{Lu2017} is repeated with our globally implicit time stepper. Two particles are placed in an unbounded shear flow, $\mathbf{u}^\infty = (y, 0)$. The rightmost particle is placed on the origin, while the leftmost particle is placed slightly above the $y$ axis at $(-8, 0.5)$. As the particles come close to each other the left particle is deflected above the right one and continues on a trajectory to the right with a constant $y$ value. The particles are discretized with 32 points, meaning that the arc length spacing is $2\pi/32 \approx 0.196$. In this experiment the particles never get within $2h$ of one another, meaning that if we take $\delta$ to be less than $2h$ the collision constraint is not necessary. 
\begin{figure}[!h]
\begin{center}
\includegraphics{figures/shear_setup1.pdf}
\begin{tabular}{c c}
\includegraphics{figures/shear_displacement1.pdf} &
\includegraphics{figures/separation_displacement1.pdf}
\end{tabular}
\end{center}
\caption{Shear experiment. Top: initial setup and trajectories of the centers of both rigid bodies. Bottom left: the $y$ coordinate of the center of the particle initially on the left as a function of its horizontal displacement. Bottom right: the final vertical displacement of the particle initially on the right as a function of separation distance.}\label{fig:shear_experiment}
\end{figure}

We can use this setup to look at effect of the contact force on the time reversibility of the flow. By reversing the shear direction at $t=10$ we can measure the relative error between the initial particle center and the particle center at $t=20$. The results for various value of $\delta$ are shown in table \ref{tab:reverse}. For $\delta = 0$ (no contact constraint) we get first order accuracy in the reversibility of the flow. This is expected as we are using forward Euler as our time stepper. For $\delta \geq 2$ we see that we no longer recover reversibility. This is due to the fact that the force has the no the effect of shifting the particles onto a streamline that is contact free. If we reverse the direction of the flow we will be traversing this contact-free streamline. 

\begin{table}[!h]
\begin{center}
\begin{tabular}{c| c c c c c}
$ $ & & & $\Delta t$ & &\\
$\delta$ & $4\e{-2}$ &$ 2\e{-2}$ & $1\e{-2}$ & $5\e{-3}$ & $2.5\e{-3}$\\
\hline
0 & $1.35\e{-1}$ & $7.32\e{-2}$ & $3.74\e{-2}$ & $2.00\e{-2}$ & $1.01\e{-2}$\\
$2h$ & $1.88\e{-1}$ & $1.41\e{-1}$ & $1.17\e{-1}$ & $1.08\e{-1}$ & $1.02\e{-1}$\\
$2.25h$ & $2.55\e{-1}$ & $2.08\e{-1}$ & $1.87\e{-1}$ & $1.78\e{-1}$ & $1.73\e{-1}$\\
$2.50h$ & $3.05\e{-1}$ & $2.69\e{-1}$ & $2.52\e{-1}$ & $2.45\e{-1}$ & $2.40\e{-1}$\\
$2.75h$ & $3.64\e{-1}$ & $3.31\e{-1}$ & $3.13\e{-1}$ & $3.07\e{-1}$ & $3.03\e{-1}$\\
$3.00h$ & $4.12\e{-1}$ & $3.88\e{-1}$ & $3.72\e{-1}$ & $3.67\e{-1}$ & $3.63\e{-1}$
\end{tabular}
\end{center}
\caption{Study of time reversibility of the shear flow setup depicted in figure \ref{fig:shear_experiment}. At $t=10$ the flow direction is reversed and we calculate the relative error in the starting position and the position at $t=20$. When the collision constraint is active and force is needed to keep the particles apart we no longer recover time reversibility.}\label{tab:reverse}
\todo[inline]{include $\delta = h$ even if there is no contact.  We
should see first order}
\end{table}


\subsection{Taylor-Green}

One of the advantages of using a globally implicit time stepper is that it can handle dense suspensions without requiring an excessively small time step. We will look at a dense suspension of 75 rigid particles in unbounded Taylor-Green flow. The background velocity field $\mathbf{u}^\infty$ is given by $(\cos(x)\sin(y), -\sin(x)\cos(y))$. Snapshots of the simulation are shown in figure \ref{fig:taylor_green}. 

\begin{figure}[!h]
\begin{center}
\begin{tabular}{c c c c}
\includegraphics[width=3cm]{figures/taylor_green0.pdf} &
\includegraphics[width=3cm]{figures/taylor_green1.pdf}&
\includegraphics[width=3cm]{figures/taylor_green2.pdf} &
 \includegraphics[width=3cm]{figures/taylor_green_tracks.pdf}
\end{tabular}
\end{center}
\todo[inline]{Finish this plot so that the green line goes right to the
time horizon}
\caption{Dense suspension in unbounded Taylor-Green flow. Particles are discretized with 32 points and a minimum separation of $0.05h$ is enforced. Snapshots of the simulation at $t=0$, $t=15$ and $t=30$ are shown. The bottom right plot shows the track of the center of the particle shaded in magenta for different time step sizes. Each line in that plot is marked where a repulsion force is necessary to enforce the minimum separation. }\label{fig:taylor_green}
\end{figure}

\subsection{Taylor-Couette}

We can use our model to investigate rheological properties of suspensions. Here we will look at suspensions of varying volume fraction and particle aspect ratio; specifically we will look at 5 and 10 percent volume fractions and ellipse shaped particles of aspect ratio 3 to 1 and 6 to 1. Initially the particles are placed randomly inside the device; the initial configurations are shown in figure \ref{fig:couette_setup}. 

\begin{figure}[!h]
\begin{center}
\begin{tabular}{c c c c}
\includegraphics[width=3cm]{figures/couette_005_3_1.pdf} & \includegraphics[width=3cm]{figures/couette_005_6_1.pdf} &
\includegraphics[width=3cm]{figures/couette_010_3_1.pdf} & \includegraphics[width=3cm]{figures/couette_010_6_1.pdf}
\end{tabular}
\end{center}
\caption{Four initial configurations for Taylor-Couette flow with varying volume fraction $\phi$ and aspect ratio $r$. From left to right: 1) $\phi=5\%$, $r = 3:1$, 2) $\phi=5\%$, $r=6:1$, 3) $\phi=10\%$, $r=3:1$, 4) $\phi=10\%$, $r=6:1$.}\label{fig:couette_setup}
\end{figure}

By measuring the torque on the inner cylinder we can get an instantaneous value of the effective viscosity of the suspension. As can be seen in figure \ref{fig:torque}, the effective viscosity increases with $\phi$, but is generally lower with the 6 to 1 shaped particles.
\begin{figure}[!h]
\begin{center}
\includegraphics{figures/couette_torque.pdf}\\
\end{center}
\caption{Instantaneous effective viscosity for the cases shown in figure \ref{fig:couette_setup}. The inner cylinder is fixed, while the outer one spins. The ratio of the torque on the inner cylinder to the baseline torque (i.e. the torque on the inner cylinder of a device with no particles) gives the instantaneous effective viscosity.}\label{fig:torque}
\end{figure} 

Another quantity we may be interested is the alignment of the particles. On average we expect the particles to align in the direction of the shear flow, which in this case is the $\hat{\theta}$ direction. Figure \ref{fig:angles} shows the deviation from this expected orientation as a function of time for the four cases. We see that the 5\% volume fractions align quicker than the 10\% volume fractions. 

\begin{figure}[!h]
\begin{center}
\includegraphics{figures/couette_angle_mean.pdf}\\
\end{center}
\caption{Deviation from the expected alignment angle. }\label{fig:angles}
\todo[inline]{Need a better way to present these results.  Problem is
that a 0 value corresponds to both no alignment and perfect alignment}
\end{figure} 

\begin{figure}[!h]
\begin{center}
\includegraphics{figures/pressure_contour.pdf}
\end{center}
\caption{Pressure distribution inside a Couette device. The outer cylinder is spinning while the inner cylinder is fixed. }\label{fig:dissipation}
\end{figure}


%%%%%%%%%%%%%%%%%%%%%%%%%%%%%%%%%%%%%%%%%%%%%%%%%%%%%%%%%%%%%%%%%%%%%%%
\subsection{Radial Flow}
\todo[inline]{Inlet with wall of grated circles}



%%%%%%%%%%%%%%%%%%%%%%%%%%%%%%%%%%%%%%%%%%%%%%%%%%%%%%%%%%%%%%%%%%%%%%%
\section{Conclusions\label{s:conclusions}}


%%%%%%%%%%%%%%%%%%%%%%%%%%%%%%%%%%%%%%%%%%%%%%%%%%%%%%%%%%%%%%%%%%%%%%%
\begin{appendices}
An appendix
\end{appendices}


\bibliographystyle{plainnat} 
\bibliography{bibliography}
\biboptions{sort&compress}
\end{document}
