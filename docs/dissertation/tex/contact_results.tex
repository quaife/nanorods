\chapter{Results}

We use our new time stepping method to simulate bounded and unbounded
suspensions of two-dimensional rigid bodies in a viscous fluid.  The
main parameters are the minimum separation distance $d_m$, the number
of discretization points of each rigid body, $N_p$, and each solid wall,
$N_w$, and the initial time step size $\Delta t$. We perform convergence
studies and investigate the effect of the STIV algorithm on the
reversibility of the flow.  To further demonstrate the consequence of
STIV, we include plots of streamlines that cross whenever the collision
detection algorithm is applied.  The particular experiments we perform
are now summarized.
\begin{itemize}
  \item {\bf Shear Flow}: We consider the standard problem of two
  identical rigid circles in the shear flow $\uu = (y,0)$ with the left
  body slightly elevated from the right body.  We report similar results
  to those presented in~\cite{Lu2017}, but we are able to take smaller
  initial displacements and minimum separation distances.  The contact
  algorithm breaks the reversibility of the flow, and this effect is
  illustrated and quantified.

  \item {\bf Taylor-Green Flow}: We simulate a concentrated suspension
  of 48 rigid ellipses in an unbounded Taylor-Green flow. At the
  prescribed separation distance, our new time stepping method is able
  to stably reach the time horizon, while the locally semi-implicit time
  integrator proposed by Lu et al.~\cite{Lu2017} results in the STIV
  algorithm stalling, even with $\Delta t = 10^{-8}$.

  \item {\bf Porous Monolayer Injection}: We consider a suspension of
  confined rigid circular bodies motivated by an experiment by MacMinn
  et al.~\cite{MacMinn2015}.  The geometry is an annulus with an inflow
  at the inner boundary and an outflow at the outer boundary.  We again
  examine the effect of contact force on the reversibility of the flow, and compute
  the shear strain rate and make qualitative comparisons to results for
  deformable bodies~\cite{MacMinn2015}.

  \item {\bf Taylor-Couette Flow}: With the ability to do high area
  fraction suspensions without imposing a large non-physical minimum
  separation distance, we simulate rigid bodies of varying aspect ratios
  inside a Taylor-Couette device.  We examine the effect of the rigid
  body shape and area fraction on the effective viscosity and the
  alignment angles.
\end{itemize}



%%%%%%%%%%%%%%%%%%%%%%%%%%%%%%%%%%%%%%%%%%%%%%%%%%%%%%%%%%%%%%%%%%%%%%%
\section{Shear Flow}
\label{sec:shear}
We consider two rigid circular bodies in the shear flow ${\uu}(\xx) =
(y,0)$.  One body is centered at the origin, while the other body is
placed to the left and above of the origin.  With this
initial condition, the particles come together, interact, and then
separate.  Both bodies are discretized with $N=32$ points and the arc
length spacing $h = 2\pi/32 \approx 0.196$.  This experiment was also
performed by Lu et al.~\cite{Lu2017}, and we compare the two time
stepping methods.

We start by considering the time step size $\Delta t = 0.4$ and
minimum separation distance $\delta = 0$ (no contact algorithm). 
Our new globally implicit method successfully reaches the time
horizon without requiring a repulsion force.  However, with the same
$\Delta t$, the local explicit time stepping results in a collision
between the bodies, so the collision algorithm is required to reach the
time horizon.  Alternatively, the time step size can be reduced, but, as
we will see, for sufficiently dense suspensions, even an excessively
small time step size results in collisions.  Next, in
Figure~\ref{fig:shear_experiment}, we investigate the effect of the
minimum separation distance on the position of the rigid bodies.  The
top plot shows the trajectory of the left body as it approaches,
interacts, and finally separates from the body centered at the origin.
In this simulation, we use our new globally implicit time integrator,
but the STIV contact algorithm is not applied.  The bottom left plot
shows the trajectory of the particle when the contact algorithm is
applied with varying levels of separation.  Notice that the trajectories
are identical until near $x=0$ when the particle separation first falls
below the minimum separation distance.  Finally, in the bottom right
plot, the final vertical displacement of the body initially on the left is
plotted.  These results are computed for the locally implicit time
stepping method~\cite{Lu2017}, and the general trend of the trajectories
are similar.
\begin{figure}[!h]
  \begin{center}
    \includegraphics{figures/shear_setup1.pdf}
    \begin{tabular}{c c}
      \includegraphics{figures/shear_displacement1.pdf} &
      \includegraphics{figures/separation_displacement1.pdf}
    \end{tabular}
  \end{center}
\caption[Displacement caused in a shear flow experiment as a function of minimum separation distance]{\label{fig:shear_experiment} Shear experiment. Top: The initial
setup and trajectory of the left body.   Bottom left: The left body's
trajectory for varying minimum separation distances.  Notice how
the trajectories are identical until shortly before $x=0$ when the
contact algorithm is first applied.  Bottom right: The final vertical
displacement of the left particle for varying minimum separation
distances.}
\end{figure}

We next investigate the effect of the collision algorithm on the time
reversibility of the flow. We reverse the shear direction at $t=10$ and
measure the error between the body's center at $t=0$ and $t=20$.  We
expect an error that is the sum of a first-order error caused by time
stepping, and a fixed constant caused by the minimum separation
distance.  The results for various values of $d_m$ are reported in
Table~\ref{tab:reverse}. The contact algorithm is not applied
when $d_m=0$ and $d_m=h$, and we observe the expected
first-order convergence.  When $d_m \geq 2h$, the bodies are
deflected onto contact-free streamlines when their proximity reaches the
minimum separation distance.  After the flow is reversed, the bodies
again pass one another, but they are now on contact-free streamlines, so
the initial deflection is not reversed.  For these larger values of
$d_m$, we see in Table~\ref{tab:reverse} that the error eventually
plateaus as $\Delta t$ is decreased indiciating that the error due to the contact algorithm dominates.

\begin{table}[!h]
\caption[Demonstration of the breaking of reversibility after repulsion forces added]{A study of time reversibility of the shear flow example. At
$t=10$, the flow direction is reversed and we calculate the relative
error in the initial and final positions. When the collision constraint
is active and force is needed to keep the bodies apart the contact algorithm dominates the error in
the reversibility.}\label{tab:reverse}
\begin{center}
\begin{tabular}{c| c c c c c}
$ $ & & & $\Delta t$ & &\\
$d_m$ & $4\e{-2}$ &$ 2\e{-2}$ & $1\e{-2}$ & $5\e{-3}$ & $2.5\e{-3}$\\
\hline
0 & $1.35\e{-1}$ & $7.32\e{-2}$ & $3.74\e{-2}$ & $2.00\e{-2}$ & $1.01\e{-2}$\\
$h$ & $1.35\e{-1}$ & $7.32\e{-2}$ & $3.74\e{-2}$ & $2.00\e{-2}$ & $1.01\e{-2}$\\
$2h$ & $1.88\e{-1}$ & $1.41\e{-1}$ & $1.17\e{-1}$ & $1.08\e{-1}$ &
$1.02\e{-1}$\\
$2.25h$ & $2.55\e{-1}$ & $2.08\e{-1}$ & $1.87\e{-1}$ & $1.78\e{-1}$ &
$1.73\e{-1}$\\
$2.50h$ & $3.05\e{-1}$ & $2.69\e{-1}$ & $2.52\e{-1}$ & $2.45\e{-1}$ &
$2.40\e{-1}$\\
$2.75h$ & $3.64\e{-1}$ & $3.31\e{-1}$ & $3.13\e{-1}$ & $3.07\e{-1}$ &
$3.03\e{-1}$\\
$3.00h$ & $4.12\e{-1}$ & $3.88\e{-1}$ & $3.72\e{-1}$ & $3.67\e{-1}$ &
$3.63\e{-1}$
\end{tabular}
\end{center}

\end{table}

The break in reversibility is further demonstrated by examining
individual streamlines.  In Figure~\ref{fig:shear_cross}, we compute the
streamline of the left body for three different initial placements.  We
set $d_m=3h$ for all the streamlines.  With this threshold, only the
bottom-most streamline falls below $d_m$.  Therefore, as the bodies approach, the streamlines behave as expected---they do
not cross.  However, when the contact algorithm is applied to the blue
streamline, the streamlines cross.

\begin{figure}[!h]
\begin{center}
\includegraphics{figures/shear_streamlines_cross.pdf}
\end{center}
\caption[Crossing of streamlines after application of repulsion forces]{\label{fig:shear_cross} The contact algorithm causes
streamlines to cross. Keeping the minimum separation fixed at
$d_m=3h$, we vary the starting $y$ location of the left body. The
teal and red streamlines do not require a repulsion force to enforce the
minimum separation between the bodies, but the blue streamline does.
Once the contact algorithm is applied, the blue streamline crosses the
other streamlines (middle inset). This crossing of the streamlines
breaks the reversibility of the simulation.}
\end{figure}


\section{Taylor-Green}

For planar flows, we can separate suspensions into dilute and
concentrated regimes by comparing the number of bodies per unit area,
$\nu$, to the average body length $\ell$. If $\nu < 1/\ell^2$, the suspension is
 dilute, otherwise it is concentrated
(in 2D planar suspensions, unlike 3D suspensions, there is no
semi-dilute regime).  We consider the suspension of 75 rigid bodies in
the Taylor-Green flow $\mathbf{u}^\infty = (\cos(x)\sin(y),
-\sin(x)\cos(y))$.  The number of bodies per unit area is $\nu \approx
3.1$ which is greater than $1/\ell^2=1.1$.  Therefore, this suspension
is well within the concentrated regime. 

We discretize the bodies with $N=32$ points and select the minimum
separation distance $d_m=0.05h$. Snapshots of the simulation are
shown in Figure \ref{fig:taylor_green}.  In this concentrated
suspension, the bodies frequently come into contact.  If
the interactions between these nearly touching bodies are treated
explicitly, this leads to stiffness.  Our time stepper controls this
stiffness by treating these interactions implicitly, and the simulation
successfully reaches the time horizon.  We performed the same
simulation, but with the locally implicit time stepping
method~\cite{Lu2017}.  Because of the near-contact, smaller time step
sizes must be taken.  We took time step sizes as small as $10^{-8}$, and
the method was not able to successfully reach the time horizon.  This
exact behavior has also been observed for vesicle
suspensions~\cite{Quaife2014}.  In the bottom right plot of
Figure~\ref{fig:taylor_green}, we show the trajectory of one body for
different time step sizes.  The dots denote locations where the contact
algorithm is applied.  For this very complex flow, the trajectories are
in good agreement with different time step sizes.

\begin{figure}[!h]
  \begin{center}
    \begin{tabular}{c c }
      \includegraphics[width=6cm]{figures/taylor_green0.pdf} &
      \includegraphics[width=6cm]{figures/taylor_green1.pdf}\\
      \includegraphics[width=6cm]{figures/taylor_green2.pdf} &
      \includegraphics[width=6cm]{figures/taylor_green_tracks_new.pdf}
    \end{tabular}
  \end{center}
  \caption[Snapshots of particles suspended in a Taylor-Green flow]{\label{fig:taylor_green} Snapshots of a dense suspension in
  an unbounded Taylor-Green flow.  The number of bodies per unit area
  $\nu$ is approximately 3.1. This is greater than $1/\ell^2 = 1.1$,
  which puts the simulation well within the concentrated regime.  Bodies
  are discretized with 32 points and the minimum separation is
  $\delta=0.05h$.  The bottom right plot shows the trajectory of the
  center of the colored body for different step sizes. Each line in that
  plot is marked where a contact force is applied to the colored body to enforce the minimum
  separation. }
\end{figure}

%%%%%%%%%%%%%%%%%%%%%%%%%%%%%%%%%%%%%%%%%%%%%%%%%%%%%%%%%%%%%%%%%%%%%%%
\section{Fluid Driven Deformation}
A recent experiment considers a dense monolayer packing of soft
deformable bodies~\cite{MacMinn2015}.  Motivated by this experiment, we
perform numerical simulations of rigid bodies in a similar device.  We
pack rigid bodies in a Couette device, but with a very small inner
boundary.  The boundary conditions are an inflow or outflow of rate $Q$
at the inner boundary with an outflow or inflow at the outer cylinder.
This boundary condition corresponds to injection and suction of fluid
from the center of the experimental microfluidic device.  In the
experimental setting, the soft bodies are able to reach the outer
boundary, and the resulting boundary condition would not be uniform at
the outer wall.  So that we can apply the much simpler uniform inflow or
outflow at the outer boundary, we force the rigid bodies to remain
well-separated from the outer wall.  We accomplish this by placing a
ring of {\em fixed} rigid bodies halfway between the inner and outer
cylinders (Figure~\ref{fig:radial}).  The spacing between these fixed
bodies is sufficiently small that the mobile bodies are not able to
pass.  Since the outer boundary is well-separated from the fixed bodies,
the outer boundary condition is justifiably approximated with a uniform
flow.

\begin{figure}[h!]
\begin{center}
\includegraphics{figures/injection_plate_setup.pdf}
\end{center}
\caption[Sketch of the geometry for a confined monolayer suspension]{\label{fig:radial} The geometry used in our numerical
experiment that is motivated by the experimental setup of MacMinn et
al.~\cite{MacMinn2015}.  The fixed solid bodies are shaded in black.}
\end{figure}

We start by examining the effect of the contact algorithm on the
reversibility of the flow.  We again reverse the flow at time $T$ and
run the simulation until time $2T$. The rigid bodies are in contact for
much longer than the shear example in Section~\ref{sec:shear}, so
maintaining reversibility is much more challenging.
Figure~\ref{fig:macminn} shows several snapshots of the simulation, and
the bottom right plot superimposes the initial and final configurations.
We observe only a minor violation of reversibility, and it is largest
for bodies that were initially near the fixed bodies---the contact
algorithm is applied to these bodies most frequently.

\begin{figure}[h!]
  \begin{tabular}{c c c}
    \includegraphics{figures/injection_plate0.pdf}&
    \includegraphics{figures/injection_plate2.pdf}&
    \includegraphics{figures/injection_plate4.pdf}\\
    \includegraphics{figures/injection_plate6.pdf}&
    \includegraphics{figures/injection_plate8.pdf}&
    \includegraphics{figures/injection_plate8_overlay.pdf}\\
  \end{tabular}
  \caption[Snapshots of a confined monolayer suspension]{\label{fig:macminn} Snapshots of a rigid body suspension
  motivated by an experiment for deformable bodies~\cite{MacMinn2015}.
  Fluid is injected at a constant rate starting at $t=0$. At $t=4$ the
  flow direction is reversed. Fixed bodies are colored in red, while
  bodies subject to a repulsion force are colored in green. The initial
  configuration has been superimposed on the final configuration at
  $t=8$ to show the effect of the repulsion forces on reversibility.}
\end{figure}

In~\cite{MacMinn2015}, the shear strain rate is measured to better
characterize the flow.  In Figure~\ref{fig:macminn_stress}, we  plot the shear strain rate for the simulation in
Figure~\ref{fig:macminn}.  A qualitative comparison of the numerical and
experimental results are in good agreement.  In particular, the
petal-like patterns in Figure~\ref{fig:macminn_stress} are also observed
in the experimental results.

\begin{figure}[h!]
  \begin{tabular}{c c c}
    \includegraphics{figures/shear_stress1.pdf}&
    \includegraphics{figures/shear_stress2.pdf}&
    \includegraphics{figures/shear_stress3.pdf}\\
    \includegraphics{figures/shear_stress6.pdf}&
    \includegraphics{figures/shear_stress5.pdf}&
    \includegraphics{figures/shear_stress4.pdf}
  \end{tabular}
  \begin{center}
    \includegraphics{figures/colorbar.pdf}
  \end{center}
  \caption[Shear strain rate for a confined monolayer suspension]{ The shear strain rate
  $\log_{10}(|\sigma_{xy}|)$  of the suspension in
  Figure~\ref{fig:macminn}.  The formation of the petal-like patterns is
  also observed experimentally for a suspension of deformable
  bodies~\cite{MacMinn2015}.}\label{fig:macminn_stress}
\end{figure}

\section{Taylor-Couette}
In many industrial applications, for example pulp and paper manufacturing, suspensions of rigid elongated fibers are encountered. Motivated by these suspensions we investigate rheological and statistical
properties of confined suspensions.  We consider suspensions of varying
area fraction and body aspect ratio; specifically we consider 5, 10,
and 15 percent area fractions and elliptical bodies of aspect ratio,
$\lambda$ of 1, 3, and 6.  In all the examples, $\nu < 1/\ell^2$, so all the
suspensions are in the dilute regime.  The bodies initial locations are
random, but non-overlapping (Figure \ref{fig:couette_setup}).  The flow
is driven by rotating the outer cylinder at a constant angular velocity
while the inner cylinder remains fixed. 


\begin{figure}[!h]
\begin{center}
\begin{tabular}{c c c c}
\includegraphics[width=3cm]{figures/couette_005_3_1.pdf} &
\includegraphics[width=3cm]{figures/couette_005_6_1.pdf} &
\includegraphics[width=3cm]{figures/couette_010_3_1.pdf} &
\includegraphics[width=3cm]{figures/couette_010_6_1.pdf}
\end{tabular}
\end{center}
\caption[Initial fiber configurations inside a Couette apparatus]{Four initial configurations for Taylor-Couette flow with
varying volume fraction $\phi$ and aspect ratio $\lambda$. From left to
right: 1) $\phi=5\%$, $\lambda= 3$, 2) $\phi=5\%$, $\lambda=6$, 3)
$\phi=10\%$, $\lambda=3$, 4) $\phi=10\%$,
$\lambda=6$.}\label{fig:couette_setup}
\end{figure}

Before measuring any rheological properties, the outer wall completes one full
revolution so that the bodies are well-mixed and
approaching a statistical equilibrium.  We start by considering the
alignment of the bodies. The alignment is particularly insightful since
many industrial processes involve fibers suspended in a flow, and the
alignment affects the material properties~\cite{Larson1999}.  One way to
measure the alignment is the order parameter, $S$, defined as,
\begin{align*}
  S = \left\langle \frac{d \cos^2\tilde{\theta} - 1}{d - 1} \right\rangle,
\end{align*}
where $d$ is the dimension of the problem ($2$ in our case),
$\tilde{\theta}$ is the deviation from the expected angle, and $\langle
\cdot\rangle$ averages over all bodies.  If $S=1$, all bodies are
perfectly aligned with the shear direction, $S=0$ corresponds to a
random suspension (no alignment), and $S=-1$ means that all bodies are
perfectly aligned perpendicular to the shear direction.  In our
geometric setup, a body centered at $(x,y)$ has an expected angle of
$\tilde{\theta}\tan^{-1}(y/x) + \pi/2$, and the average alignment of the
bodies will be in the direction of the shear, which is also
perpendicular to the radial direction.

Since the initial condition is random, the initial configurations in
Figure~\ref{fig:couette_setup} have an order parameter $S\approx 0$. As
the outer cylinder rotates, we see in Figure~\ref{fig:angles} that $S$
increases rapidly. The area fractions $\phi$ we consider have a
minor effect on $S$; however, the aspect ratio has a large effect.  In
particular, suspensions with slender bodies align much better with the
flow.

\begin{figure}[!h]
\begin{center}
\includegraphics{figures/order_parameter.pdf}\\
\end{center}
\caption[Simulated order parameters for fibers inside a Couette apparatus]{ The order parameter of different fiber
concentrations and aspect ratios. We see that the 6:1 fibers align
better. The 6:1 fibers rotate through the angle perpendicular to
the shear direction more quickly than the 3:1 fibers and thus spend more
time approximately aligned with the shear direction. The dashed lines
represent the order parameter for a suspension in an unbounded shear
flow with bodies that do not interact hydrodynamically. The red line
shows $\lambda=3$, while the blue line shows $\lambda=6$. }\label{fig:angles}
\end{figure} 

This matches the known dynamics of a single body in an unbounded shear
flow, where the body will align with the shear direction on
average. Bodies with a high aspect ratio rotate quickly when then they are
perpendicular to the shear direction and spend more time nearly aligned
with the shear direction. We compare our results to the time averaged
order parameter of a single elliptical body in an unbounded shear flow.
If the shear rate is $\dot{\gamma}$, a single elliptical body rotates with period $\tau
= \pi/(2| \dot{\gamma}|)(\lambda + \lambda^{-1})$~\cite{Jeffery1922}
according to 
\begin{align*}
  \varphi(t) ~=~ \tan^{-1}\left(\frac{1}{\lambda}\tan\left(
    \frac{\lambda \dot{\gamma}t}{\lambda^2 + 1}\right)\right).
\end{align*}
The time average order parameter is then,
\begin{align*}
  \langle S\rangle ~=~ \frac{1}{\tau}\int_0^\tau\left( 
    2\cos^2(\varphi(t)) - 1\right)~\text{d}t ~=~ \frac{\lambda -1}{\lambda+1}.
\end{align*}
Independent of the shear rate, for $\lambda= 3$ the theoretical $\langle
S\rangle$ is 1/2 and for $\lambda=6$ it is  5/7. Table \ref{tab:order}
shows the time and space averaged order parameter for the Couette
apparatus. We see that in all cases our computed time averaged order
parameter is higher than the theoretical single fiber case. This could
be due to the hydrodynamic interactions between the bodies, or the
effect of the solid walls.

In the absence of solid walls and hydrodynamic interactions between
bodies, a suspension will align and disalign. The period of the order
parameter in this case is the same as the rotational period for a single
fiber. In Figure \ref{fig:angles} the theoretical order parameter is
shown for a suspension of non-hydrodynamically interacting fibers in an
unbounded shear flow.  Hydrodynamic interactions prevent the suspension
from disaligning completely.



\begin{table}[!h]
\caption[Time averaged order parameter in a Couette apparatus]{The time averaged order parameter during the second revolution of the Couette apparatus. The higher aspect ratio fibers align better on average. The alignment is in all cases higher than predicted for a single Jeffery orbit.
}\label{tab:order}
\begin{center}
\begin{tabular}{c |c |c |c}
area fraction, $\phi$ & aspect ratio, $\lambda$ & computed $ \langle S \rangle$ & theoretical
$\langle S \rangle$ (single fiber)\\
\hline
5\%  & 3 & 0.52 & 0.50 \\
10\% & 3 & 0.60 & 0.50 \\
15\% & 3 & 0.65 & 0.50 \\
5\%  & 6 & 0.91 & 0.71 \\
10\% & 6 & 0.89 & 0.71
\end{tabular}
\end{center}

\end{table} 

Another quantity of interest in rheology is the effective viscosity of a
suspension. The shear viscosity $\mu$ relates the bulk shear stress
$\sigma_{xy}$ of a Newtonian fluid to the bulk shear rate
$\dot{\gamma}$, 
\begin{align*}
  \sigma_{xy} = \mu\dot{\gamma}.
\end{align*}
Adding bodies increases the bulk shear stress of a suspension.  The
proportionality constant relating the increased $\sigma_{xy}$ to the
shear rate is the {\em apparent viscosity}, and the ratio between the
apparent viscosity and the bulk viscosity is the effective viscosity
$\mu_{\text{eff}}$.  Experimentally, the bulk shear stress is often
computed by measuring the torque on the inner cylinder~\cite{Koos2012}.
Numerically, this is simply the strength of the rotlet corresponding to the
inner cylinder.  By computing the ratio of the torque on the inner
cylinder with bodies to the torque without bodies we determine the
effective viscosity of a suspension.  Figure~\ref{fig:torque} shows the
effective viscosity increases with $\phi$, but is generally lower for
bodies with aspect ratio $\lambda=6$. This is because higher aspect ratio
bodies align themselves better, and thus contribute less to the bulk
shear stress. The spikes in \ref{fig:torque} occur when a repulsion
force is added to the system. Similar spikes were observed 
in Lu \emph{et al.}~\cite{Lu2017}. To smooth the results we use a multiscale local polynomial transform to smooth the data shown in Figure \ref{fig:torque}.
\begin{figure}[!h]
\begin{center}
\includegraphics{figures/couette_torque_smooth_005.pdf}\\
\includegraphics{figures/couette_torque_smooth_010.pdf}\\
\includegraphics{figures/couette_torque_smooth_015.pdf}
\end{center}
\caption[Simulated bulk viscosity for a Couette apparatus]{Instantaneous bulk effective viscosity for various volume
fractions and body aspect ratios. The inner
cylinder is fixed, while the outer one rotates at a constant angular velocity. The transparent lines represent the raw data, while the solid lines have been smoothed using a multiscale local polynomial transform. }\label{fig:torque}
\end{figure} 

Finally, instead of computing the instantaneous effective viscosity,
experimenters are interested in the time averaged effective viscosity of
a suspension. In Table~\ref{tab:viscosity}, we report the average
instantaneous effective viscosity over the second revolution of the
outer cylinder
\begin{table}[!h]
\caption[Time averaged effective viscosity in a Couette apparatus]{Time averaged effective viscosity for various area fractions
and aspect ratios. The time average is done between the first and second
revolutions of the outer cylinder. As $\phi$ increases the effective
viscosity increases as expected. In general higher aspect ratio bodies
increase the viscosity less than lower aspect bodies. }\label{tab:viscosity}
\begin{center}
\begin{tabular}{c| c | c |  c}
$\lambda$ & 5\% area fraction  & 10\% area fraction  & 15\% area fraction\\
\hline
1 & 1.12 & 1.22 & 1.42 \\
3 & 1.10 & 1.23 & 1.36 \\
6 & 1.08 & 1.18 & 
\end{tabular}
\end{center}

\end{table}


%\begin{figure}[!h]
%\begin{center}
%\includegraphics{figures/pressure_contour.pdf}
%\end{center}
%\caption{Pressure distribution inside a Couette device. The outer cylinder is spinning while the
%inner cylinder is fixed. }\label{fig:dissipation}
%\end{figure}