\chapter{Introduction}

Many natural and man-made materials can be classified as particle suspensions. The occurrence of suspensions spans many fields, including biofluids, foodstuff, and manufacturing processes.  To give three examples, blood is a suspension of platelets and red and white blood cells, mayonnaise is a suspension of oil drops, and pulp used in paper manufacturing is a suspension of fibers \cite{Larson1999}. In all of these examples, hydrodynamics plays a critical role. In particular, the particles are not only carried by the flow, but affect the flow. This is different from suspensions involving granular media, for example atmospheric pollutants, where the particle interactions are negligible, and do not affect the flow. This dissertation focuses on viscous suspensions of non-deformable rigid bodies such as fibers. 

Rigid body suspensions in viscous fluids are important for manufacturing composite materials, where rigid fibers are distributed throughout the material. The orientation and distribution of these fibers can be used to tune the mechanical, thermal, or electrical properties of the composite. For example, the composite is strongest in the direction of maximum fiber alignment, and weakest in the direction of minimum fiber alignment~\cite{Folgar1984}. Operations such as injection molding, extrusion, or casting are used to process molten suspensions. The liquid suspension is then solidified, holding the fibers in place. Once the fiber orientations are known, mathematical models can predict the mechanical properties of the suspension \cite{Advani1987}. Modeling the fibers in the liquid sate is important because the distribution and orientation of the fibers depend strongly on the flow history of the melt during processing. In addition, the rheological properties of the suspension, which govern the flow, depend themselves on the suspended fibers. Thus there exists a relationship between the particle distribution and orientations and the fluid flow. 

The study of  suspensions has a long history. In his 1906 Ph.D. thesis, Einstein \cite{Einstein1906} calculated the effective viscosity of a dilute suspension of spherical particles in a fluid of viscosity $\mu$. He found that if the suspension is dilute enough so that hydrodynamic interactions play no role, the effective viscosity of the suspension is
\[ \mu_{\text{eff}} = \mu(1 + 2.5\phi),\]
where $\phi$ is the particle volume fraction. This relationship is only valid for very dilute suspensions, those with $\phi < 0.05$. For suspensions with a higher volume fraction, hydrodynamic interactions between particles are important and higher-order terms in $\phi$ are needed to estimate the effective viscosity. For an extensional flow, Batchelor \cite{Batchelor1971} computed the extensional viscosity of a suspension of spheres assuming pair interactions are dominant,
\[ \mu_{\text{eff}} = \mu(1 + 2.5\phi + 6.95\phi^2).\]
 Even in this relatively simple case, the shear viscosity of a suspension of spheres demonstrates shear-thickening, a non-Newtonian behavior.  
 
The contribution of non-spherical particles to the viscosity of a suspension is more complicated. 
 In a shear flow, for instance, non-spherical rigid bodies contribute less to the effective viscosity when they are aligned with the flow direction. Thus their contribution is not static and their dynamics must be considered. The motion of a single ellipsoid fiber in shear flow was analyzed by Jeffery~\cite{Jeffery1922}. In $\mathbb{R}^2$, an ellipse with length $\ell$ and diameter $d$ suspended in a shear flow with shear rate $\dot{\gamma}$ rotates with period $(\pi/\dot{\gamma})(\lambda + \lambda^{-1})$, where $\lambda$ is the aspect ratio $\ell/d$. The period of the ``Jeffery orbit'' increases with $\lambda$, and the amount of time the fiber spends  aligned with the flow direction also increases with $\lambda$.  Given suitable corrections, the trajectories of fibers of different elongated shapes, such as cylinders, can also be described as Jeffery orbits. 

The concentration of the fibers plays a critical role in the properties of rigid body suspension. We quantify the concentration of a suspension of fibers by defining $\nu$ to be the average number of fibers per unit volume. If $\nu < 1/\ell^3$, the suspension is in the dilute regime and interactions between fibers are rare. In this case the fiber trajectories can be quantitatively described  by Jeffery orbits. 

Outside the dilute regime, interactions between fibers are more frequent and must be considered. To define the orientation of a fiber, we assign to each fiber a unit vector $\pp$ that points in the direction of its semi-major axis (Figure \ref{fig:pp}).  In the semi-dilute regime, $1/\ell^3 < \nu < 1/d\ell^2$, Batchelor related the average stress tensor $\bm{\sigma}$ to the distribution of fiber orientations $\pp$ and the rate of strain tensor $\ee = (\nabla\uu + (\nabla\uu)^T)/2$.  
Letting $\Psi(\pp,\xx)$ be the probability of a fiber at location $\xx$ being oriented in the direction $\pp$, we define the ensemble averages
\[a_{ij} =\int p_i p_j \Psi(\pp,\xx)~\text{d}\pp, \qquad a_{ijk\ell}(\xx) = \int p_i p_j p_k p_{\ell}\Psi(\pp,\xx)~\text{d}\pp.\]
In the absence of Brownian motion and assuming purely hydrodynamic interactions between fibers (i.e. no external forces and torques), and a slender body approximation ($\lambda \gg 1$), the stress tensor can be approximated as
\begin{equation}\label{eq:stress_folgar} \sigma_{ij} = 2\mu e_{ij} + \nu\zeta a_{ijk\ell} e_{k\ell},\end{equation}
where $\zeta$ is a drag coefficient that depends on the size and concentration of the fibers and the fluid viscosity~\cite{Batchelor1971}.

\begin{figure}[!h]
\begin{center}
\includegraphics{figures/orientation_vector.pdf}
\end{center}
\caption[Sketch of the orientation vector]{Sketch of the orientation vector $\pp$.}\label{fig:pp}
\end{figure}
Computational simulations of fiber suspensions can be either implicit or explicit. Implicit models treat the suspension as a continuum, while explicit models track each fiber individually. In an implicit approach, a suitable fluid model (Navier-Stokes, Stokes) uses the stress tensor \eqref{eq:stress_folgar} and combines it with the Folgar-Tucker model  \cite{Folgar1984, Jack2006} that governs the evolution of $a_{ij}$ 
\[ \frac{D}{D t}a_{ij} = \frac{1}{2}\Omega_{ik}a_{kj} + \frac{1}{2}a_{ik}\Omega_{kj} + \frac{1}{2}\lambda\left(e_{ik}a_{kj} + a_{ik}e_{kj} - 2e_{k\ell}a_{ijk\ell}\right) + C_I ||\ee||(2\delta_{ij} - 6a_{ij}).\]
Here $\bm{\Omega} = (\nabla \uu - (\nabla\uu)^T)/2$ is the vorticity tensor and $C_I$ is an empirically derived interaction coefficient that depends on the volume fraction. The limitations of this approach are:
\begin{itemize}
	\item it requires closure relations and fiber interaction coefficients based on empirical data,
	\item all the rods must be uniform and slender,
	\item the model is invalid in the concentrated regime.
\end{itemize}

In the concentrated regime, $\nu >1/d\ell^2$, the interactions between fibers dominate and the contribution to the total stress can only be computed by adding the force from each fiber individually \cite{Ausias2006, Lindstroem2008}. To do this, explicit models are required. Explicit models are also useful in the dilute and semi-dilute regimes to model irregularly shaped bodies, or to compute interaction constants and tensor closures to be used in the Folgar-Tucker model. The goal of this dissertation is to develop techniques that are capable of accurately and stably simulating dense suspensions of rigid bodies. This method will be computationally efficient, robust, and will resolve rigid bodies of arbitrary shape over a wide range of concentrations.

\section{Related Work}

Many explicit models represent rigid bodies as prolate ellipsoids~\cite{Ausias2006}, sets of connected beads~\cite{Joung2001, Yamamoto1996}, rods \cite{Lindstroem2007, Schmid2000}, or slender bodies~\cite{Fan1998,Tornberg2006, Batchelor1970a}. Among explicit methods, we can define two general groups. The first models the fluid phase and updates the bodies by computing the velocity of the fluid on the surface of the particles. The immersed boundary method \cite{Mittal2005}, level set methods \cite{Dou2007}, lattice Boltzmann methods \cite{Ladd1994a, Ladd1994b}, smoothed particle hydrodynamics \cite{Polfer2016}, and dissipative particle hydrodynamics \cite{Pivkin2010} all fall within this category. An advantage of these methods is that fluid inertia effects can be included by modeling the fluid equations with the Navier-Stokes equations. Moreover, if appropriate, the solvent can be non-Newtonian. A disadvantage of explicit methods is that it requires computing the fluid velocity at all points inside the fluid domain, something that adds significant computational expense, but may not be of interest. In addition, these methods often struggle with resolving unbounded or periodic domains. 

In problems where the fluid is Newtonian and inertia is negligible, the problem simplifies considerably. In this case, the fluid equations become the Stokes equations, and potential theory \cite{Ladyzhenskaya1963} allows us to compute the rigid body velocities without solving for the fluid velocity inside the fluid domain. This allows us to determine the motion of the bodies solely based on the configuration of the particles and the background flow. Stokesian dynamics and boundary integral equations both make use of potential theory to compute the motions of the particles without directly computing the velocity in the bulk solvent. 

As mentioned, the hydrodynamic interactions between bodies must be considered when simulating suspensions. If two bodies are well-separated, their hydrodynamic interactions are well-approximated with a multipole expansion of a few terms. This means that the hydrodynamic interactions of well-separated bodies are almost independent of the body's shape. For nearly touching bodies, however, the interaction is much stronger and depends strongly on the body's shape. That said, if the bodies have simple shapes, for example spheres or ellipsoids, the interactions of nearly touching bodies can be precomputed using lubrication theory. Stokesian dynamics \cite{Brady1988, Claeys1993} combines the multipole representation for bodies that are far apart with lubrication theory to compute interactions between nearly touching bodies. This has been shown to be very effective for a wide range of problems, including those with small Peclet numbers where Brownian motion needs to be considered.

For general shaped rigid bodies however, lubrication theory is not computationally tractable, since the number of possible hydrodynamic interactions of nearly touching bodies that would need to be precomputed is loo large. Instead of Stokesian dynamics, boundary integral equations (BIEs) can be used. BIEs reduce the Stokes equations to an integral over the boundary of the rigid bodies (and solid walls). This leads to a dimension reduction when compared to explicit methods that discretize the entire fluid domain. Since BIEs require only a discretization of the boundary, they can naturally resolve problems with complicated geometries, including unbounded ones. In the context of the Stokes equations, solutions derived from BIEs automatically satisfy the  incompressibility condition and the far field condition. This is in contrast to many other methods, for example the finite element method.  BIEs have been successfully applied to various kinds of suspensions, including rigid bodies~\cite{Bystricky2018,Corona2017, Tornberg2006}, vesicles \cite{Quaife2014, Quaife2015, Rahimian2010}, and drops \cite{Sorgentone2018}. In addition, BIEs can be rigorously analyzed using the Fredholm alternative and the spectral theorem. This theory guides numerical methods whose convergence properties are provably optimal. Of course, the BIE formulation must be approximated by applying discretization methods. One discretization technique applies quadrature to a collocation scheme. By using appropriate quadrature, high-order or even spectral accuracy can be achieved. The limitation of BIEs is that the resulting linear system is dense. In contrast, methods that discretize the entire domain result in larger, sparse linear systems. Nonetheless, by using efficient iterative solvers and fast summation techniques, an $\mathcal{O}(N)$ solver for the dense linear system is possible, where $N$ is the number of discretization points on the boundary of the domain.

Power and Miranda \cite{Power1987, Power1993} developed an integral equation representation for the Stokes equations. Their formulation is particularly nice since its discretization is high-order, invertible, and compatible with fast algorithms so that it can be efficiently solved. In particular, the condition number of the discrete linear system is independent of the size of the system $N$. The Power and Miranda representation has been used to simulate suspensions of rigid bodies \cite{Bystricky2018,Corona2017, Tornberg2006}. 

The Stokes equations prohibit contact between bodies in finite time, however, numerical errors may cause overlaps between bodies. When using a model that explicitly represents rigid bodies, care must be taken when advancing in time to avoid collisions and overlaps. An algorithm that prevents collisions between rigid bodies, but allows for a large time step size is desirable. Adaptive time stepping and local refinement can help \cite{Kropinski1999, Quaife2016}, but particularly in concentrated suspensions, adaptive time stepping may require an excessively small time step to ensure contact is avoided. 

Repulsion forces \cite{Flormann2017, Liu2006, Lu2017} are another approach to prevent contact. Popular forces are variants of a Morse or Lennard-Jones potential that grows as a high-order polynomial as two bodies approach \cite{Flormann2017, Liu2006}. Spring based repulsion forces \cite{Kabacogulu2017, Zhao2013} have also been used. Methods such as these are inherently heuristic and require the choice of tuning parameters. Furthermore, they introduce stiffness, and therefore still require a small time step to maintain stability. Lastly, they still do not guarantee that collisions are avoided. We adopt a method outlined in \cite{Lu2017} that guarantees that each time step is collision-free. This method enforces a constraint on the variational form of the Stokes equations. This constraint appears as a repulsion force in the resulting Euler-Lagrange equations (i.e. the regular Stokes equations), and is free of tuning parameters and guarantees that contact is avoided.
 
\section{Contributions}

Lu \emph{et al.} \cite{Lu2017} use a locally-implicit time stepper coupled with a contact algorithm to simulate suspensions of vesicles and rigid bodies. Locally-implicit time steppers treat the inter-particle hydrodynamic interactions explicitly by lagging them from the previous time step. This method yields a block diagonal system to solve at every time step. Unfortunately, due to nearly touching bodies, the locally-implicit time stepper requires a small step size or a large minimum separation distance to maintain stability. In this dissertation, we couple the contact algorithm \cite{Lu2017} with a globally-implicit time stepper. Globally-implicit time steppers treat all hydrodynamic interactions implicitly. This results in a dense linear system to solve at every time step. While each globally-implicit time step is more computationally expensive than a locally-implicit time step, it allows larger time steps and a smaller minimum separation distance between particles. In particular, for certain problems, the locally-implicit time stepper is not stable, even for time steps on the order of $10^{-7}$. In contrast, globally-implicit time steppers are stable for an acceptable time step size, a characteristic that is necessary to simulate dense suspensions.

We use this method to investigate concentrated suspensions, alignment angles, and the effective viscosity of suspensions in confined geometries. Compared to previous methods, our method better resolves the interactions between closely touching bodies, and the result is a stable time stepping method that avoids contact. We investigate the effect of the repulsion force on the time reversibility of the suspension and demonstrate that these repulsion forces cause the particles to jump streamlines and therefore break reversibility. In addition, we make qualitative comparisons between our method and other methods and experiments from the literature. 

\section{Limitations}

The main limitation of this model is that it is  developed in two dimensions. To be useful for composite manufacturing, a three-dimensional model is required. Three-dimensional models for boundary integral equations are well-developed \cite{Mammoli2006, Corona2017 }, though they are more difficult to implement and computationally more expensive. Moreover, they require many more unknowns, meaning that far fewer particles can be simulated in three dimensions at a fixed number of discretization points. 

To compute the shear viscosity of a suspension, we have chosen to model the torque exerted on a Couette apparatus. In reality though, wall effects have a large impact on the particle trajectories. For this reason, unbounded periodic simulations, perhaps involving Lees-Edwards boundary conditions \cite{Lees1972}, are preferred. In the context of BIEs, this boundary condition has not been investigated, however periodic simulations involving boundary integral equations have been developed \cite{Klinteberg2014,Marple2015}.

The computation of repulsion forces requires the solution of a nonlinear complementarity problem (NCP). This NCP is linearized and this results in a sequence of linear complementarity problems (LCPs). From a mathematical standpoint, there is no theory that guarantees that these LCPs has a unique solution. This might cause the NCP iterations to stall or diverge, which is something that we observe for concentrated suspensions in bounded domains. In addition, even if the LCPs do have a unique solution, many LCP iterations may be required to converge to the NCP solution. Since each LCP solution requires us to solve the Stokes equations, this can be very computationally expensive. 

